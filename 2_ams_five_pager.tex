% !TeX program = pdfLaTeX
\documentclass[12pt]{article}
\usepackage{amsmath}
\usepackage{graphicx,psfrag,epsf}
\usepackage{enumerate}
\usepackage{natbib}
\usepackage{textcomp}
\usepackage[hyphens]{url} % not crucial - just used below for the URL
\usepackage{hyperref}

%\pdfminorversion=4
% NOTE: To produce blinded version, replace "0" with "1" below.
\newcommand{\blind}{0}

% DON'T change margins - should be 1 inch all around.
\addtolength{\oddsidemargin}{-.5in}%
\addtolength{\evensidemargin}{-.5in}%
\addtolength{\textwidth}{1in}%
\addtolength{\textheight}{1.3in}%
\addtolength{\topmargin}{-.8in}%

%% load any required packages here



% tightlist command for lists without linebreak
\providecommand{\tightlist}{%
  \setlength{\itemsep}{0pt}\setlength{\parskip}{0pt}}




\begin{document}


\def\spacingset#1{\renewcommand{\baselinestretch}%
{#1}\small\normalsize} \spacingset{1}


%%%%%%%%%%%%%%%%%%%%%%%%%%%%%%%%%%%%%%%%%%%%%%%%%%%%%%%%%%%%%%%%%%%%%%%%%%%%%%

\if0\blind
{
  \title{\bf On enhancing knowledge transfer from a contingent
workforce}

  \author{
        Roel Peters \\
    Antwerp Management School\\
      }
  \maketitle
} \fi

\if1\blind
{
  \bigskip
  \bigskip
  \bigskip
  \begin{center}
    {\LARGE\bf On enhancing knowledge transfer from a contingent
workforce}
  \end{center}
  \medskip
} \fi

\bigskip
\begin{abstract}

\end{abstract}

\noindent%
{\it Keywords:} 
\vfill

\newpage
\spacingset{1.45} % DON'T change the spacing!

Traditionally, management consultancy came in the form of strategic
advice. However, over the past decades, the business world has witnessed
that organizations increasingly focus on their core business, while
outsourcing everything else to a third party or a ``contingent
workforce'': consultants and freelancers \citep{msp2022}. This practice
has even penetrated organizations' core value chain, a phenomenon known
as business process outsourcing or BPO \citep{shi1}. However, working
with a contingent workforce, entails considerable risk.

\hypertarget{relevancy}{%
\section{Relevancy}\label{relevancy}}

\begin{enumerate}
\def\labelenumi{\arabic{enumi}.}
\tightlist
\item
  Trend: external \textgreater{} internal
\item
  Price of consultants
\item
  Hot topic: \emph{when} to rely on consultants?
\item
  Productivity: is it worth it? If so, under what circumstances?
\item
  Job market \& macro-economic aspects
\end{enumerate}

\hypertarget{state-of-the-literature}{%
\section{State of the literature}\label{state-of-the-literature}}

\hypertarget{what-is-it-consultancy}{%
\subsection{What is IT consultancy?}\label{what-is-it-consultancy}}

\hypertarget{definitions}{%
\subsubsection{Definitions:}\label{definitions}}

\begin{itemize}
\tightlist
\item
  Contingent workforce
\item
  Freelance
\item
  Digital consultancy
\item
  IT consultancy
\item
  Management consultancy
\item
  Outsourcing
\item
  Knowledge-intensive business services
\item
\end{itemize}

\hypertarget{emergence}{%
\subsubsection{Emergence}\label{emergence}}

See \citep[ 120-130]{armbruster2006}

\hypertarget{why-digital-consultancy-a-practical-perspective}{%
\subsubsection{Why digital consultancy? A practical
perspective}\label{why-digital-consultancy-a-practical-perspective}}

\citet{turner1982} provides a hierarchy of consulting purposes. The
first five are traditionally associated with consultancy, while the last
three are seen as by-products, and often not as explicit goals.

\begin{enumerate}
\def\labelenumi{\arabic{enumi}.}
\tightlist
\item
  Provide requested information
\item
  Provide solution to given problem
\item
  Conduct diagnosis that may redefine problem
\item
  Provide recommendations
\item
  Assist implementation. This is not without controversy, as
  traditionally, some argued that ``one who helps put recommendations
  into effect takes on the role of manager and thus exceeds consulting's
  legitimate bounds.'' Also, ``a frequent dilemma for experienced
  consultants is whether they should recommend what they know is right
  or what they know will be accepted.''
\item
  Build consensus and commitment
\item
  Facilitate client learning
\item
  Improve organizational effectiveness
\end{enumerate}

If we go from consultancy, in general, to IT consultancy, it is
essential to understand that over the past decades, their possible roles
and variety of responsibilities have expanded drastically.
\citet[20-25]{swanson2010} has described five different ways how
consultants can contribute to an organization's innovation process
through IT.

\begin{itemize}
\tightlist
\item
  \emph{Business strategy}: IT consultancy can lead the organization to
  new pursuits and technologies they wouldn't have discovered
  themselves. Second, IT consultancy can frame the need for innovation
  in strategic terms, and they prepare and legitimize the need for
  change.
\item
  \emph{Technology assessment}: IT consultancy can facilitate the
  comprehension of IT technologies and its alternatives.
\item
  \emph{Business process improvement}: Innovations that involve IT
  usually come to fruition only after business processes have been
  revamped. Business process changes usually require an outside-in view
  and offer rich opportunities for consulting.
\item
  \emph{Systems integration}: In many cases, introducing a new
  technology requires that it needs to be integrated with existing
  systems and users need to be onboarded. This type of IT consultancy
  usually requires coding skills, hands-on design and implementation
  expertise
\item
  \emph{Business support services}: Finally, once the implementation is
  completed, it can take a while before the solution is entirely
  assimilated. IT consultants can provide complementary IT services such
  as support and maintenance until the technology is entirely embedded
  in the organization.
\end{itemize}

See also \citep{bessant1995}.

In their 1994 study, for which they interviewed over 100 decision,
\citet[10-17]{lacity1994} group expectations with regards to outsourcing
into four categories: financial, business, technical and political
expectations.

\hypertarget{financial-expectations}{%
\subsubsection{Financial expectations}\label{financial-expectations}}

Reducing costs Improving cost controls Restructuring IT budgets

\hypertarget{business-expectations}{%
\subsubsection{Business expectations}\label{business-expectations}}

Focusing on core activities Facilitating mergers \& acquisitions
Starting-up a company

\hypertarget{technical-expectations}{%
\subsubsection{Technical expectations}\label{technical-expectations}}

Improving technical service Accessing talent \& technologies

\hypertarget{political-expectations}{%
\subsubsection{Political expectations}\label{political-expectations}}

\hypertarget{knowledge-transfer-diffusion}{%
\subsubsection{Knowledge transfer \&
diffusion}\label{knowledge-transfer-diffusion}}

Return to \citet{turner1982}.

Something about knowledge transfer here \citep{sturdy2009}.

Nevertheless, there are constraints to knowledge transfer. According to
\citet[128-129]{cohen1990}, ``the ability to evaluate and utilize
outside knowledge is largely a function of the level of prior related
knowledge {[}such as{]} basic skills, or even a shared language but may
also include knowledge of the most recent scientific or technological
developments in a given field. {[}\ldots{]} These abilities collectively
constitute what we call a firm's \emph{absorptive capacity}.''

There is substantial research on knowledge management (as a
multidisciplinary discipline within the field of information science)
and knowledge transfer (as a broad topic within the discipline).
Furthermore, there seems to be some academic interest in knowledge
transfer in a principal-agent context \citep{ning2008, haines2003}, as
is the case with between an organization and their contingent workforce.
This research could be key in steering and narrowing the scope of the
research.

\hypertarget{why-digital-consultancy-a-theoretical-perspective}{%
\subsection{Why digital Consultancy: a theoretical
perspective}\label{why-digital-consultancy-a-theoretical-perspective}}

\textless Unsure if this part should be expanded, or even in the final
paper\textgreater{} \citet[8-10]{nevo2007} outlines five theoretical
frameworks that contribute to the question why firms should (not)
utilize external IT capabilities.

\begin{itemize}
\tightlist
\item
  The \emph{resource-based view} (RBV) claims that firms can earn
  sustainable above-normal returns by possessing rare and valuable
  resources and that they have isolating mechanisms that prevent the
  dissemination of those resources.
\item
  The \emph{micro-economic} view assumes that firms operate in fully
  competitive markets and, contrary to RBV, that above-normal returns
  are competed away by rivals or new entrants. Utilizing IT and external
  capabilities is simply a matter of delivering the optimal quantity of
  products and their prices.
\item
  \emph{Transaction cost economics} claims that firms are interested in
  identifying areas in which they can outperform market-based
  interactions and do themselves. The commodity-like nature of IT
  solutions make them very prone to being outsourced to consultants.
\item
  \emph{Institutional theory} does not reject the micro-economic theory
  that firms try to make rational choices, but acknowledges that there
  are cognitive and rational constraints. Consequently, the choice for
  internal or external capabilities involves trust, relationships,
  personal beliefs and aspirations.
\item
  \emph{Identification theory}: Individuals derive value and meaning
  from group membership. This can impact the involvement of external
  actors for IT projects in a negative way.
\end{itemize}

See also \citep[ 66]{armbruster2006}

\hypertarget{transaction-cost-economics}{%
\subsubsection{Transaction cost
economics}\label{transaction-cost-economics}}

Intro on Ronald Coase.

\citet[16-17]{nevo2007} concludes that his research supports the
transaction cost hypothesis: ``when the internal IT capability is weak,
developing and implementing an IT solution is likely to cost more
compared with hiring external IT consultants to do the same job.''
Furthermore, the reverse situation also supports the identification
theory: ``IT consultants will not receive the legitimacy they require
{[}\ldots{]} if their knowledge and expertise do not differ from that
possessed by the in-house IT team. Under these circumstances, external
IT consultants' impact on IT productivity is expected to be lower.''

Nevertheless, there is a serious limitation in the work of
\citet{nevo2007}. An IT project is assumed to be fixed in time, with
fixed parameters. It does not account for ``vendor learning''
\citep{wu2004} during the project. \citet{cha2009} tackles this
shortcoming with a model built around two parameters:

\begin{enumerate}
\def\labelenumi{\arabic{enumi}.}
\tightlist
\item
  (production) knowledge transfer rate: the ability of the client to
  capture knowledge from the vendor.
\item
  (coordination) knowledge depreciation rate: the ability of the client
  to retain coordination knowledge as it outsources IT activities.
\end{enumerate}

They conclude that firms with a low production knowledge transfer rate
(e.g.~unmotivated employees) should insource or outsource all their IT
capabilities. When they also have a high coordination knowledge
deprecation rate (e.g.~bad project management), insourcing is the only
option. On the other hand, when both the production knowledge transfer
and coordination knowledge depreciation rate are high, the optimal rate
of of IT outsourcing is also high.

See also \citep[ 12-14]{armbruster2006}, \citep{canback1999}.

\hypertarget{resource-based-view}{%
\subsubsection{Resource-based View}\label{resource-based-view}}

In \citet[177-180]{willcocks2003}, four types of sourcing options for
developing IT projects are outlined, of which three involve consultants.

\begin{enumerate}
\def\labelenumi{\arabic{enumi}.}
\tightlist
\item
  Internal development: has the the advantage of internalization of the
  learning outcomes, but comes with high costs related to mistakes and
  being the first mover.
\item
  Outsourcing: has the advantage of tapping into existing knowledge and
  experience, and the ability to get quickly up to speed. However,
  internalization of learning outcomes is not guaranteed, and
  consultants may not be familiar with existing organizational
  processes. For example, the development of an internal application by
  an external party.
\item
  Insourcing/partnering: has the same advantages as outsouring, with the
  added bonus of facilitating the internalization of the learning
  outcomes. The disadvantage is mostly related to a more complex project
  management, with a variety of parties involved. For example: long-term
  contracts with IT consultants who operate side-by-side with an
  organization's own staff.
\item
  Cheap-sourcing: when IT projects are low risk, and far from the core
  business, and organization should consider cheap-sourcing. This option
  involves low investments and effort, but also comes with no internal
  learning. For example: development of a new promotional website by a
  digital agency.
\end{enumerate}

In the same research paper, \citet[188-189]{willcocks2003} identify two
congruent four-quadrant matrices to assess sourcing options.

\begin{itemize}
\tightlist
\item
  By business activity: non-critical, commoditized applications should
  be out-sourced. Critical, commoditized applications should be
  insourced or built in-house, and differentiating, critical
  applications should be built in-house or acquired.
\item
  By market comparison: A high-cost, low-quality market leads to
  in-house development, while a high-cost, high-quality market should
  lead to insourcing. A low-cost, low-quality market leads to
  cheap-sourcing and a low-cost, high-quality market is perfect for
  outsourcing.
\end{itemize}

\hypertarget{identification-theory}{%
\subsubsection{Identification Theory}\label{identification-theory}}

The research by \citet[311-313]{schwarz2005} claims that it matters
\emph{who} implements an IT project: ``technology-enabled inertia can be
explained through understanding an employee's social identifications and
his or her associated cognitions, where inertia exists on a sliding
scale of change.'' By defending their self-image, low-status groups can
hinder the implementation of an application. The sourcing assessment
needs to incorporate this finding.

\hypertarget{embeddedness-theory}{%
\subsubsection{Embeddedness Theory}\label{embeddedness-theory}}

\citep[ 14-16]{armbruster2006}

\hypertarget{sociological-neoinstitutionalism}{%
\subsubsection{Sociological
neoinstitutionalism}\label{sociological-neoinstitutionalism}}

\citep[ 8-11]{armbruster2006}

\hypertarget{problem-statement}{%
\section{Problem statement}\label{problem-statement}}

\hypertarget{quality-problems-adverse-selection}{%
\subsection{Quality problems \& Adverse
Selection}\label{quality-problems-adverse-selection}}

\citep[ 69-75]{armbruster2006} outlines various reasons for quality
uncertainty and groups them in two categories.

Category 1: Formal institutional uncertainty

\begin{itemize}
\tightlist
\item
  Consulting is an unbounded profession.
\item
  Consulting is an unbounded industry.
\item
  Consulting has unbounded service lines and product standards.
\end{itemize}

Category 2: Transactional Uncertainty

\begin{itemize}
\tightlist
\item
  Confidentiality
\item
  Product intangibility
\item
  Interdependent cooperation
\end{itemize}

The economic barriers to entry \citep[ 463]{fee2004} in IT consultancy
(and consultancy in general) are few to none. Anyone with experience in
a specific field, sector or technology can wrap it as advice and sell it
to whoever wants to hear it. Furthermore, the author of this paper is
unaware of legal barriers to entry.

According to some, assessing the quality of consultants is impossible.
For example, according do \citet[40]{bloomfield1995}, ``there can be no
presumed separation between technical skills and political skills, and
no ranking between the two in terms of their importance for consultancy
practice and the development of IT in user organizations.'' Furthermore,
\citet[101-102]{bettencourt2002} states that for knowledge-intensive
business services (KIBS) to succeed, a lot depends on the client.
``Client co-production roles {[}\ldots{]} are emergent, multi-faceted,
and highly collaborative because clients themselves possess much of the
knowledge and competence that a KIBS firm needs to successfully deliver
its service solution.''

For evaluating (future) performance of consultants, one has to rely on
informal and relational criteria \citep[ 277]{wright2002}. According to
\citet[250]{clark1993}, ``the main trust-producing mechanism
{[}\ldots{]} is the `closed' social structure; a form of individual
trust. The formal, institutional-based, trust-producing mechanisms are
weak. It is the contractual guarantees, and the history of past
transactions underlying reputation, which overcome the potential effects
of adverse selection and moral hazard.''

Especially in a situation where past transactions are absent,
``management consultancies must convey in some way to their clients that
they have something valuable to offer. {[}\ldots{]} consultants are able
to take control of the process by which impressions and perceptions of
their service are created. By managing the creation of these images
consultants are able to persuade clients of their value and quality.
Management consultancies are therefore `systems of persuasion' \emph{par
excellence} excellence and impression management is not external to the
core of their work but is at its core.'' \citep[ 35]{clark1998}

In the existing literature, these remarks are part of a critical
paradigm regarding consultants \citep[ 4-5]{armbruster2006}. Authors
point to the contestable nature of consulting, the self-interest of
consultancy firms, and the stretching of consultancy advice.

See also:

\begin{itemize}
\tightlist
\item
  \citep{wright2002}
\item
  \citep{david2013}
\item
  \citep{mahoney2016}
\item
  \citep{nayyar1990}
\end{itemize}

\hypertarget{moral-hazard}{%
\subsection{Moral Hazard}\label{moral-hazard}}

See \citep[ 72-73]{armbruster2006}.

\hypertarget{potential-solutions}{%
\subsection{Potential solutions}\label{potential-solutions}}

Several mechanisms have already been proposed.
\url{https://link.springer.com/chapter/10.1057/9780230362994_12}

\hypertarget{psychological-contract-obligations}{%
\subsubsection{Psychological contract
obligations}\label{psychological-contract-obligations}}

According to \citet[357]{ang2004}, the legal interpretation of an IT
outsourcing contract is too limited. Instead, they claim that the
construct of a \emph{psychological contract} is more appropriate for
analyzing the relationship between an IT service supplier and customer.
The strength of psychological contract theory is threefold:

\begin{enumerate}
\def\labelenumi{\arabic{enumi}.}
\tightlist
\item
  it focuses on mutual obligations;
\item
  the emphasis is on psychological obligations;
\item
  the emphasis is on the individual level--not on the organizations as
  parties of the contract.
\end{enumerate}

Consequently, the psychological contract not only comprises the legal
contract, but also the unwritten promises, interpersonal relations, and
the individual interpretations and perceptions. Since consultancy
contracts can become extremely complex (with project descriptions going
into the ten thousands of words), and the involved parties entangled in
multiple ways, these intangible aspects can gain prominence. The
research in \citet[369-70]{ang2004} outlines several psychological
contract obligations that positively impact the success of an outsourced
IT project.

\begin{itemize}
\tightlist
\item
  On the supplier side: (1) clear authority structures, (2) knowledge
  transfer by educating the customer, (3) building inter-organizational
  teams.
\item
  On the customer side: (1) clear specification of requirements, (2)
  prompt payment, and (3) project ownership and monitoring.
\end{itemize}

Closely related is the work by \citet[9-13]{willcockskern} that makes a
distinction between the contractual level and the cooperative level. The
contractual level is about payment for the exchange of services and the
transfer of assets, information \& consultants. The cooperative level
involves formal communication mechanisms; personal investments in time,
resources \& knowledge; mutual goals \& objectives and social bonds. The
atmosphere surrounding the former is heavily impacted by developments at
the latter. A respondent in \citet[9]{willcockskern} states that ``the
contract is a bit like a nuclear deterrent. You need one and you have
got to have a framework, but if you've got to use it you are probably in
trouble.''

\hypertarget{networked-reputation}{%
\subsubsection{Networked Reputation}\label{networked-reputation}}

See \citep[ 75]{armbruster2006}.

\hypertarget{public-reputation}{%
\subsubsection{Public Reputation}\label{public-reputation}}

See \citep[ 76]{armbruster2006}

\hypertarget{experience-based-trust}{%
\subsubsection{Experience-based trust}\label{experience-based-trust}}

See \citep[ 76-77]{armbruster2006}, \citep{clark1993}

\hypertarget{research-questions}{%
\section{Research Questions}\label{research-questions}}

\begin{enumerate}
\def\labelenumi{\arabic{enumi}.}
\tightlist
\item
  Why do Belgian firms rely on digital consultants?
\item
  Which control mechanisms do Belgium firms have in place with regards
  to adverse selection of digital consultants?
\item
  Which control mechanisms positively impact success of engaging with a
  consultancy firm?
\end{enumerate}

\bibliographystyle{agsm}
\bibliography{references.bib}


\end{document}
