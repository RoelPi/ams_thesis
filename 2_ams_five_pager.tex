% !TeX program = pdfLaTeX
\documentclass[12pt]{article}
\usepackage{amsmath}
\usepackage{graphicx,psfrag,epsf}
\usepackage{enumerate}
\usepackage{natbib}
\usepackage{textcomp}
\usepackage[hyphens]{url} % not crucial - just used below for the URL
\usepackage{hyperref}

%\pdfminorversion=4
% NOTE: To produce blinded version, replace "0" with "1" below.
\newcommand{\blind}{0}

% DON'T change margins - should be 1 inch all around.
\addtolength{\oddsidemargin}{-.5in}%
\addtolength{\evensidemargin}{-.5in}%
\addtolength{\textwidth}{1in}%
\addtolength{\textheight}{1.3in}%
\addtolength{\topmargin}{-.8in}%

%% load any required packages here



% tightlist command for lists without linebreak
\providecommand{\tightlist}{%
  \setlength{\itemsep}{0pt}\setlength{\parskip}{0pt}}




\begin{document}


\def\spacingset#1{\renewcommand{\baselinestretch}%
{#1}\small\normalsize} \spacingset{1}


%%%%%%%%%%%%%%%%%%%%%%%%%%%%%%%%%%%%%%%%%%%%%%%%%%%%%%%%%%%%%%%%%%%%%%%%%%%%%%

\if0\blind
{
  \title{\bf (working title) Control mechanisms for working with
consultants}

  \author{
        Roel Peters \\
    Antwerp Management School\\
      }
  \maketitle
} \fi

\if1\blind
{
  \bigskip
  \bigskip
  \bigskip
  \begin{center}
    {\LARGE\bf (working title) Control mechanisms for working with
consultants}
  \end{center}
  \medskip
} \fi

\bigskip
\begin{abstract}

\end{abstract}

\noindent%
{\it Keywords:} 
\vfill

\newpage
\spacingset{1.45} % DON'T change the spacing!

\hypertarget{introduction}{%
\section{Introduction}\label{introduction}}

\hypertarget{relevancy}{%
\section{Relevancy}\label{relevancy}}

\begin{enumerate}
\def\labelenumi{\arabic{enumi}.}
\tightlist
\item
  Trend: external \textgreater{} internal
\item
  Price of consultants
\item
  Hot topic: \emph{when} to rely on consultants?
\item
  Productivity: is it worth it? If so, under what circumstances?
\item
  Job market \& macro-economic aspects
\item
  Recent news in BE/NL
\end{enumerate}

\hypertarget{defining-digital-consultancy}{%
\section{Defining ``Digital
Consultancy''}\label{defining-digital-consultancy}}

``Digital consultancy'' is defined as consultancy in (either or both)
technological and organizational aspects of digital transformation. The
following two sections substantiate this definition by elaborating on
the two concepts that comprise this definition. First, six properties
that define ``consultancy'' are discussed, followed by an elaboration on
``digital transformation.''

\hypertarget{consultancy}{%
\subsection{Consultancy}\label{consultancy}}

Consultants, or management consultants, have been described through a
multitude of metaphors and nicknames: ``capitalism's commissars''
\citep[ 93]{thrift2005}, ``shadowy figures operating in the background
but exercising considerable influence'' \citep[ 31]{kipping2012},
``agents of a modern rationalistic and universalistic culture'' \citep[
190]{kipping2012}, ``institutionally approved agents'' \citep[
193]{kipping2012}, ``magical figures, shamans or witch doctors'' \citep[
68]{fincham2002}, ``puppet masters'' \citep[ 69]{fincham2002} and simply
``The Big Con'' \citep{mazzucato2023}.

Their work makes use of a vague body of knowledge described as elusive,
fuzzy, perishable, indeterminate, esoteric, fluid and changeable by
\citet{muzio2011}. Furthermore, consultancy is marked by very low
professionalization as occupational entry is unprotected, the supply of
labor is unregulated and there is no formal accreditation. \citep[
20]{fincham2006} Practicising ``consultancy'' is the main criterion of
membership with competences and `time spent in the industry' as the main
differentiators.

From these remarks, one could say that who they are, and what they do is
extremely hard to describe. \citet[24]{kipping2012} states that
``definitions of management consultancy are problematic because the
permeable boundaries of the industry have resulted in significant shifts
over time in the composition of the industry. This means that what
comprises consulting work is dynamic, ever shifting, and contested as
new firms enter the industry and techniques deemed formerly appropriate,
change. Although the industry is characterized by periodic structural
shifts, at its heart it is an advisory activity built on the
client--consultant relationship. {[}\ldots{]} it is perhaps this
chimeral ability to avoid precise definition and to be able to
constantly reinvent its core services to meet ever changing
understandings of the problems that beset contemporary organizations,
which partly underpins its growing economic importance.''

The following definition of ``consultancy'' is used throughout this
paper: consultancy is a service offered by an external service provider.
Although the responsibilities of a consultant are highly contingent on
the client organization and consultancy can take many forms and require
a variety of expertise, its goals is to establish change in the
procedures, organizational structure or tools of a client organization.
Finally, the success of a consultancy engagement is often determined by
the interactivity between a consultancy firm and the client
organization.

The sections below unpack this definition and elaborate on the six
properties that it comprises.

\hypertarget{external}{%
\subsubsection{External}\label{external}}

\citet[138]{chowdhury2021} describes consultants as ``external advisors
to corporations, nonprofits, governments, and any other forms of
organizations.''

\hypertarget{change}{%
\subsubsection{Change}\label{change}}

Clearly, the construct of a ``consultant'' cannot be described by the
topic that they work on, nor their academic and professional background,
accreditation or membership. Instead, we should look at their goal(s):
\citet[1]{werr1986} implies that there is always a change process
between clients and consultants. This is confirmed by
\citet[12]{kipping2000} who states that ``management consultancies earn
money through changing current procedures in client organizations.''
Although this change is often described by consultants as a `tailored
solution', consultants provide a service, which is inherently intangible
(compared to the `solid' nature of products) and hard to evaluate
\citep[ 348]{fincham1999}, especially because an evaluation should not
only account for the content of the changes, but also in terms of the
competence development of the client, as a result of the change process
\citep[ 17]{werr1986}.

Although consultants' goal is to establish change in an organization,
their role is often symbolic. As external consultants, associated with
their ``quest for knowledge'' and their ``quest for excellence'', they
are \citep[ 9-13]{pellegrin2006} well-equipped for legitimizing hard
decisions, signaling importance and providing meaning.

Establishing change is what sets consultancy apart from temporary
staffing. ``A temp is generally not supposed to change the work
practices at the client organization. A consultant is often expected to
do just that, or at least to provide an alternative point of view.''
\citep[ 5]{furusten2000}

The aspect of change is also important for drawing boundaries between
consultancy and outsourcing. Outsourcing is the practice of obtaining
goods or services from an external provider, as a substitute for
sourcing it internally \citep[ 2]{lacity2012}. Or in the words of
\citet[374]{zhu2001}: ``the process of transferring the responsibility
for a specific business function from an employee group to a
non-employee group.'' While early IT outsourcing initiatives were rather
``total'' \citet{willcocks1995} in nature, outsourcing individual
business functions is a more common activity \citep[ 377]{zhu2001}, and
focus has shifted from cost saving to quality, productivity, flexibility
and technological diversity \citep[ 185]{kirilov2012}. This implies that
outsourcing, unlike consultancy, is not about changing a procedure or
service, but rather ensuring their continuation, by a third-party
provider.

\hypertarget{contingent}{%
\subsubsection{Contingent}\label{contingent}}

Being a consultant implies taking on a variety of responsibilities
throughout a certain time span. Although many consultants have
structured methodologies, which are converging across the industry
\citep[ 17]{werr1986}, ``{[}c{]}onsultants operate in an intense
environment that regularly entails new challenges.'' \citep[
138]{chowdhury2021} What entails consultancy work is dynamic, ever
shifting and contested with every new firm entering the industry and new
methodologies claiming the spotlight \citep[ 24]{kipping2002}.

This is what sets external consultancy apart from internal consultancy.

\hypertarget{relational}{%
\subsubsection{Relational}\label{relational}}

The nature of consultancy is often relational. First of all, a
consultant's work is embedded in an organization's web of interpersonal
relations. ``{[}T{]}he context, terms of reference, and ensueing
recommendations pertaining to a consultancy engagement may represent a
continuation, by other means, of ongoing processes of co-operation,
struggle and conflict between organizational groups.''
\citep{bloomfield1995}

Furthermore, a consultant's deliverable is often co-created with the
client. \citet[290-297]{nikolova2009} describes the client-consultant
relationship within a `social learning model'. It starts from the belief
that there is no ``knowledge out there'', and client and consultant need
to work closely together to develop problem solutions. The role of the
consultant is that of a ``facilitator of diagnosis and problem-solving''
and coach, while the client is the actual problem solver.
\citet[22]{clark1998} even goes so far as to claim that ``Like a bottle
of wine, a restaurant meal, or a book, the quality of a management
consultancy service is determined during enactment/consumption. This
indicates that the outcome of a consultancy service is highly dependent
upon the quality of the interaction between the client and the
consultant.''

A consultant needs to develop skills in order to fit in and adapt their
skill set to the needs of a client. When a relationship does not
succeed, no authority is placed on the skills of the service provider
\citep[ 10]{furusten2000} and the assignment might turnout to be
unsuccessful.

The importance of the relational aspect is what sets consultancy apart
from outsourcing \citep[ 171-173]{kipping2012}. The outcome of a
consultancy assignment often depends heavily on the interaction between
the client and the service provider. On the other hand, outsourcing
focuses more on technical capabilities and implies an integral handover
of a (set of) service(s) to an external provider that becomes the sole
responsible for delivering them.

\hypertarget{two-sided}{%
\subsubsection{Two-sided}\label{two-sided}}

Besides their tasks at the client, consultants also face internal
pressures from their employers in terms of optimized resource
utilization (billabillity), using proprietary knowledge and ``proximity
that they can develop with the client.'' \citet[138]{chowdhury2021}

``Consultant-assisted IS projects differ from internal and outsourced IS
projects, in two important respects. First, the joint project team
consists of members from client and consulting organizations that may
have conflicting goals and incompatible work practices. Second, close
collaboration between the client and consulting organizations is
required throughout the course of the project.''
\citet[255]{liberatore2010}

\hypertarget{diverse}{%
\subsubsection{Diverse}\label{diverse}}

Within this group, however, we can identify consultancy types: strategy
consulting, tax consulting, HR consulting, risk \& regulatory
consulting, etc. However, \citet[71-72]{armbruster2006} argues that the
boundaries between consultancy service types are blurred. A single
project often requires multiple types of services, but the distinction
is often artificial. Especially the boundary between strategy and IT
consultancy is opaque due to the fact that the big accounting \&
strategy firms entered the IT consultancy market to conduct
all-encompassing projects where strategy and IT meet.

\hypertarget{digital-transformation}{%
\subsection{Digital Transformation}\label{digital-transformation}}

\citet[28]{bloomfield1995} describes IT consultants as intermediaries:
``they interpose themselves between IT and clients, or between IT
suppliers and clients, in effect seeking to speak for technology. Put
another way, they seek to portray them selves as obligatory passage
points. {[}\ldots{]} the problem of choosing a particular functional
system is translated into a problem of choosing the best expert
advice.''

However, the offering by many different types of organizations of some
kind of IT consultancy for selecting, implementing, configuring and
preaching IT solutions has led to a blurring of consultancy work
\citetext{\citealp[ 31]{bloomfield1995}; \citealp[ 162]{kipping2012}}.
While consultants working for hardware and software vendors are claimed
to be motivated by the sale of their own products, consultancy firms aim
to equal themselves as suppliers of objective business advice.

These blurring lines are the result of the fact that there is more than
a technical dimension to IT solutions. Consequently, consultants frame
IT solutions not just from their technical dimension, but from their
organizational dimension, as well \citep[ 24-25]{bloomfield1995}. Like
strategy, technology, often depicted as neutral and separate from social
or political matters, can be wielded for political purposes. However,
the boundary between the merely technological and political is flexible:
a social or political problem can be translated as a technical one.

In accordance with this interpretation, the services of multinational
consultancy firms are defined or classified as consultancy in digital
operations (PwC); digital commerce \& engineering (Accenture); digital
transformation (EY, Bain, Deloitte, Tata); digital (McKinsey, KPMG);
digital, technology \& data (BCG); digitalization (Capgemini), digital
solutions (BoozAllenHamilton) and digital experience (Cognizant). Ergo,
it is remarkable that many research describes this group of consultants
as ``IT consultants''
\citep{nevo2007, loh1992, fincham2006, armbruster2006, bloomfield1995, schwarz2005},
while none of the big consultancy firms offer ``IT consultancy''.

Mindful of these findings, this paper trades in the concept of ``IT
consultancy'' for ``digital consultancy'', and defines is as
\emph{consultancy in (either or both) technological and organizational
aspects of digital transformation}. In this definition ``digital
transformation'' refers to the expectation that the use of digital
technology will lead to favorable business outcomes \citep[
104-118]{wessel2020}, by redefining or supporting the value proposition
of an organization and imposing changes on the work practices of its
organizational members. This rolls up the dichotomy between the concepts
of ``digital transformation'' and ``IT-enabled organizational
transformation'' \citep{wessel2020} into a single term, mainly for the
goal of simplicity.

\hypertarget{emergence-of-digital-consultancy}{%
\section{Emergence of Digital
Consultancy}\label{emergence-of-digital-consultancy}}

Some papers to explore:

\begin{itemize}
\tightlist
\item
  \citep[ 120-130]{armbruster2006}.
\item
  \citep{kipping2003}
\item
  \citep{kipping2012}
\item
  \citep[ 336]{fincham1999}
\item
  \citep{mckenna2006}
\end{itemize}

Three phases can be identified.

In the first phase, consultants were working with the first commercially
usable computers of the 1950s, as companies were exploring the benefits
of using them in their operations. IT consultancy emerged in three
groups of companies: technology companies, accounting and auditing
firms, and management consultancy firms. \citep[ 162]{kipping2012}

During the second phase, in the 1970s, managers began to employ IT to
align internal processes with their organizations' business objectives,
clearly pointing to an alignment of IT and strategy. The result was a
growing demand for IT consultants with a background in strategy.
Furthermore, as new application such as ERPs, CRMs, and accounting
software hit the market, IT became recognized as a facilitator of
change. IT consultants were working on client's operations and often
served as change managers.

The third phase arrived with the introduction of network computing, as
the internet fundamentally transformed the nature of commercial
transactions. This sparked the rise of a host of `dot.com consultancies'
that provides advice on how to exploit these new opportunities. This
third phase differs from the first two as it is mainly driven by new
applications and services, and not by new hardware developments. It is
during this phase that many IT consulting \& outsourcing services became
standardized, leading to rapid commoditization and lower prices. New
companies began to offer these services on a global scale, often driven
by a relatively cheap labor force in emerging economies.

\hypertarget{why-digital-consultancy-a-practical-perspective}{%
\section{Why digital consultancy? A practical
perspective}\label{why-digital-consultancy-a-practical-perspective}}

See \citet{lacity1994} for more body in the following paragraphs.

\citet{turner1982} provides a hierarchy of consulting purposes. The
first five are traditionally associated with consultancy, while the last
three are seen as by-products, and often not as explicit goals.

\begin{enumerate}
\def\labelenumi{\arabic{enumi}.}
\tightlist
\item
  Provide requested information
\item
  Provide solution to given problem
\item
  Conduct diagnosis that may redefine problem
\item
  Provide recommendations
\item
  Assist implementation. This is not without controversy, as
  traditionally, some argued that ``one who helps put recommendations
  into effect takes on the role of manager and thus exceeds consulting's
  legitimate bounds.'' Also, ``a frequent dilemma for experienced
  consultants is whether they should recommend what they know is right
  or what they know will be accepted.''
\item
  Build consensus and commitment
\item
  Facilitate client learning
\item
  Improve organizational effectiveness
\end{enumerate}

If we go from consultancy, in general, to IT consultancy, it is
essential to understand that over the past decades, their possible roles
and variety of responsibilities have expanded drastically.
\citet[20-25]{swanson2010} has described five different ways how
consultants can contribute to an organization's innovation process
through IT.

\begin{itemize}
\tightlist
\item
  \emph{Business strategy}: IT consultancy can lead the organization to
  new pursuits and technologies they wouldn't have discovered
  themselves. Second, IT consultancy can frame the need for innovation
  in strategic terms, and they prepare and legitimize the need for
  change.
\item
  \emph{Technology assessment}: IT consultancy can facilitate the
  comprehension of IT technologies and its alternatives.
\item
  \emph{Business process improvement}: Innovations that involve IT
  usually come to fruition only after business processes have been
  revamped. Business process changes usually require an outside-in view
  and offer rich opportunities for consulting.
\item
  \emph{Systems integration}: In many cases, introducing a new
  technology requires that it needs to be integrated with existing
  systems and users need to be onboarded. This type of IT consultancy
  usually requires coding skills, hands-on design and implementation
  expertise
\item
  \emph{Business support services}: Finally, once the implementation is
  completed, it can take a while before the solution is entirely
  assimilated. IT consultants can provide complementary IT services such
  as support and maintenance until the technology is entirely embedded
  in the organization.
\end{itemize}

See also \citep{bessant1995}.

In their 1994 study, for which they interviewed over 100 decision,
\citet[10-17]{lacity1994} group expectations with regards to outsourcing
into four categories: financial, business, technical and political
expectations.

\hypertarget{financial-expectations}{%
\subsection{Financial expectations}\label{financial-expectations}}

Reducing costs: - ``executives wanting to exercise control over the
management and investment of IT, but lacking the expertise.'' \citep[
233]{sturdy1998} - ``Other sources of cost reductions are the
elimination of large fixed costs during recessions and the transfer of
adjustment costs to the outsources when a new technology is adopted.''
\citep{aubert1996} * Improving cost controls * Restructuring IT budgets

\hypertarget{business-expectations}{%
\subsection{Business expectations}\label{business-expectations}}

\begin{itemize}
\tightlist
\item
  Focusing on core activities
\item
  Facilitating mergers \& acquisitions
\item
  Starting-up a company
\end{itemize}

\hypertarget{technical-expectations}{%
\subsection{Technical expectations}\label{technical-expectations}}

\begin{itemize}
\tightlist
\item
  Improving technical service
\item
  Accessing talent \& technologies
\end{itemize}

\hypertarget{political-expectations}{%
\subsection{Political expectations}\label{political-expectations}}

``\ldots using the `objectivity' and/or status of consultants to
legitimate or influence a course of action.'' \citep[ 233]{sturdy1998}

\hypertarget{knowledge-transfer-diffusion}{%
\subsection{Knowledge transfer \&
diffusion}\label{knowledge-transfer-diffusion}}

In \citet[53]{werr2002}, a manager describes how they are often caught
up in day-to-day activities, and consultants can help them take a loot
at the ``big picture'', from a strategic perspective: they make sense of
the manager's organization in relation to its environment, such as its
competitors.

Return to \citet{turner1982}.

Something about knowledge transfer here \citep{sturdy2009}.

Nevertheless, there are constraints to knowledge transfer. According to
\citet[128-129]{cohen1990}, ``the ability to evaluate and utilize
outside knowledge is largely a function of the level of prior related
knowledge {[}such as{]} basic skills, or even a shared language but may
also include knowledge of the most recent scientific or technological
developments in a given field. {[}\ldots{]} These abilities collectively
constitute what we call a firm's \emph{absorptive capacity}.''

\citet[84]{fincham2002} made an interesting observation:
organization-specific knowledge and expert knowledge are very complexly
related, and knowledge transfer can only happen into a ``well-prepared
ground.'' For this reason, consultants tend to focus their efforts on
the kinds of relationship in which a manager is relieved of their local
knowledge.

``Managers thus viewed consultants as a way of bypassing the knowledge
filters created by the organizational hierarchy, as well as the effects
of organizational politicals, which became salient in times of
reorganization and change. Management consultants were seen as a way for
managers to gain a `true' picture of what was going on in their
organizations.'' \citep[ 54]{werr2002}

``House consultants also had accumulated a unique understanding of the
client company's historical legacies, having a much longer time
perspective than individual managers who frequently changed jobs. The
consultants were thus described as the `organizational memory' of the
organization.''

\hypertarget{attracting-capabilities}{%
\subsection{Attracting capabilities}\label{attracting-capabilities}}

\begin{itemize}
\tightlist
\item
  ``Attracting capabilities that are in short supply.''
  \citet[52]{aubert1996}
\item
  ``\ldots lacking the skills for a project or, less explicitly, to
  compete with each other.'' \citep[ 233]{sturdy1998}
\item
  ``The consultants chased us so that we really implemented the change.
  'In supporting the realization of change projects, consultants
  provided methodology as well as an energizing example with their own
  style of working.'' \citep[ 54]{werr2002}
\end{itemize}

\hypertarget{why-digital-consultancy-a-theoretical-perspective}{%
\section{Why digital Consultancy: a theoretical
perspective}\label{why-digital-consultancy-a-theoretical-perspective}}

According to \citet[3-6]{armbruster2006}, the theoretical perspectives
on consultancy can be broken down into two main categories and
corresponding streams of literature. The first one is the functionalist
view, which sees consultants as ``carriers and transmitters of
management knowledge.'' The second perspective argues that the
functionalist perspective is to narrow in scope to grasp consulting
projects: client-consultants interactions are open to distortions, and
understanding them requires research. This is known as the critical
view.

\hypertarget{transaction-cost-economics}{%
\subsection{Transaction cost
economics}\label{transaction-cost-economics}}

Transaction costs economics sees economic organization as a problem of
contracting, i.e.~organizing economic activity. The starting point is
that every transaction comes with certain costs, both ex ante and ex
post. Ex ante transaction costs come in the form of drafting and
negotiating an agreement, which can become extremely complex when lots
of contingencies are present. Ex post transaction costs, on the other
hand, include maladaptation costs, when the delivery of a good or
service drifts from its initial conceptualization, the haggling costs
for adapting the contract, the setup and running of governance
structures to handle disputes and finally and recurring bonding costs to
secure commitments.

Their bounded rationality makes it impossible for humans to estimate
both the costs and risks of complete contracts, or even enacting and
enforcing them. \citep[ 53]{aubert1996} The result is that the
contractual partners often decide to leave room for adaptation and
interpretation, which, in turn, increases the risk of opportunistic
behavior (infra).

According to transaction cost economics, the decision whether a service
should be conducted in-house or purchased in the market is based on the
comparison of the sum of production and transaction costs \citep[
12]{armbruster2006}.

\citet[31]{canback1998} does a solid job explaining the role of
transaction costs in explaining the raison d'être of consultants. ``As
companies strive to reduce the production costs by exploiting scale and
scope economies, they must specialise -- which in turn leads to a need
for internal coordination. If transaction costs did not exist, then the
largest company in each market would also be the most profitable
company, since coordination between functions could be achieved without
effort. But because of transaction costs, this does not happen.'' The
result is that blue-collar jobs disappear as production costs are
reduced, while white-collar jobs, aimed at coordination, do not.

Mindful of this evolution, the assumption is that there is a high demand
for advice and (IT) solutions that improve coordination within and
between firms. These are services in which consultants are particularly
well-versed. The question to ask here is: are the transaction costs for
working with external consultants lower than for working with internal
consultants when it comes to knowledge production?

\citet[37-44]{canback1998} argues it does, and arguments from the three
critical dimensions of transactions, a popular research topic within
transaction cost economics.

\emph{Asset specificity} describes the degree to which physical, human
or site assets have a specific usage and can they not be put to use for
another purpose. With highly idiosyncratic transactions, no vendor is
willing to tailor his product or service to one client, and face
downward price pressure, since the latter acts as a monopsonist \citep[
218-228]{robinson1969}.

According to \citet[250-253]{williamson1979} higher asset specificity
leads either to one of two forms of ``relational contracting''. The
first form is bilateral governance in which there are ``admissible
dimensions for adjustment such that flexibility is provided under terms
in which both parties have confidence.'' The second form is unified
governance (i.e.~internalization or vertical integration), in which
``adaptations can be made in a sequential way without the need to
consult, complete, or revise interfirm agreements. Where a single
ownership entity spans both sides of the transactions, a presumption of
joint profit maximization is waranted.''

\citet[95-96]{williamson1985} identifies 4 types of asset specificity:

\begin{enumerate}
\def\labelenumi{\arabic{enumi}.}
\tightlist
\item
  Site specificity: the degree to which the successive stages of
  production are in close proximity to each other.
\item
  Physical asset specificity: the degree to which the physical
  properties of the product are unique.
\item
  Human asset specificity: the degree to which the skills, or
  configuration of skills within a team, are unique to an organization's
  production process.
\item
  Dedicated assets: ???
\end{enumerate}

The second dimension is the \emph{frequency} of transactions on the
buyer side. A transaction with high asset specificity does not require a
different contracting approach, because there is no subsequent phase in
which the buyer can leverage his monopsony power and stray from the
initial contract. However, when the frequency goes beyond a single
transaction ``idiosyncratic transactions are ones for which the
relationship between buyer and supplier is quickly thereafter
transformed into one of bilateral monopoly.'' \citep[
241]{williamson1985}

\emph{Uncertainty}. Within the context of transaction cost economics,
\citet[38]{shin2003} states that ``many empirical studies show mixed and
contradictory results against what transaction cost economics predicts,
especially for the concept of uncertainty.'' and as a solution, reduces
the concept to ``behavioral uncertainty'', hereby ignoring environmental
uncertainty \citep[ 391-392]{watjatrakul2005} This is in line with
\citet[79]{williamson1985} who claims that ``The proposed match of
governance structures with transactions considers only two of the three
dimensions for describing transactions: asset specificity and frequency.
The third dimension, uncertainty, is assumed to be present in sufficient
degree to pose an adaptive, sequential decision problem. {[}\ldots{]}
Since continuity now matters, {[}\ldots{]} uncertainty makes it more
imperative to organize transactions within governance structures that
have the capacity to `work things out.'\,''

When we bring these three dimensions together, we conclude that
behavioral uncertainty arises from asset specificity because it may lead
to opportunism, but only in recurrent transactions. In this context,
behavioral uncertainty and can't be disentangled from asset specificity.
\citet[78]{williamson1985} is fully aware that this is a departure from
Coase's transaction cost rationale.

To drive back the theory to the subject of consulting,
\citet[37]{canback1998} argues that it's mainly human asset specificity
that favor using consultants, since their assignments typically have a
low human asset specificity so that the solution or advice can be
reproduced at many organizations.\footnote{Furthermore, Canback claims
  that transaction frequency and uncertainty are less of an influence.
  By referring to market uncertainty, not only does he obscure the fact
  that consultants rather thrive in a context with high complexity and
  uncertainty, he also misrepresents the uncertainty dimension that is
  central in transaction cost economics. This is a prime example of the
  vagueness surrounding the concept of uncertainty in transaction cost
  economics (supra).} \citet[408]{watjatrakul2005} put the theory to the
test and compared the transaction cost view with the resource-based view
(infra) for describing the sourcing decisions in three cases and comes
to the following conclusion: ``a high-specificity asset has a major
impact on sourcing decisions. It overpowers the effect of uncertainty.''

Focusing on low-specificity assets allows consultancy firms to achieve
economies of scale. That's why they rather shun highly idiosyncratic
assignments. Rather, they'll focus on (often high-level) organizational
advice and IT architectures, since these have the biggest adaptive
properties.

Borrowing rhetoric from the resource-based view (infra)
\citet[498]{mata1995} applies Canback's conclusion on technical IT
skills: ``While technical skills are essential in the use and
application of IT, they are usually not sources of sustained competitive
advantage. {[}\ldots{]} they are usually not heterogeneously distributed
across firms. Moreover, even when they are heterogeneously distributed
across firms, they are typically highly mobile. {[}\ldots{]} firms
without the required analysis, design, and programming skills required
to make an IT investment can hire technical consultants and
contractors.'' Ergo, IT capabilities are very likely to be outsourced.

\citet[16-17]{nevo2007} also concludes that his research supports the
transaction cost hypothesis: ``when the internal IT capability is weak,
developing and implementing an IT solution is likely to cost more
compared with hiring external IT consultants to do the same
job.''\footnote{the reverse situation also supports the identification
  theory: ``IT consultants will not receive the legitimacy they require
  {[}\ldots{]} if their knowledge and expertise do not differ from that
  possessed by the in-house IT team. Under these circumstances, external
  IT consultants' impact on IT productivity is expected to be lower.''}

\hypertarget{agency-theory}{%
\subsection{Agency Theory}\label{agency-theory}}

Agency theory is very compatible with transaction cost economics. When
an organization decides to not develop a specific service in-house, and
instead, buys it in the market, it encounters agency issues: information
asymmetries and goal incompatibility. See \citet{shapiro2005}. The
organization is interested in a timely roll-out of a quality solution
for a problem they have. The consultancy firm, on the other hand, is
driven by profit maximization.

\citet{fincham2002} made an interesting addition to this typical
principal-agent problem. It starts from the premise that since managers
are agents of the owners of capital, a consultants can be described as
``an agent's agent'', extending the management's own agency function. In
other words, consultants operate at the ``outer reaches of corporate
power'', stretching corporate authority to its limits. Consequently,
their legitimacy is often problematic within the corporation that
engages with them.

Another valid point is raised by \citet[23]{basu2011} , who discovered
that pre-qualification efforts displayed little effect on adverse
selection because the consultant might present an excellent reputation,
but individual consultants assigned to a project might lack the required
skills.

\hypertarget{resource-based-view}{%
\subsection{Resource-based View}\label{resource-based-view}}

The resource-based view rejects the traditional microeconomic
assumptions that goods or services are homogeneous. Instead, it argues
that they are heterogeneously distributed across firms, and not
perfectly transferable \citet[392]{watjatrakul2005}. These resources
come in the form of assets, capabilities or organizational processes.
Firms can obtain above-normal results if they can establish a
competitive advantage by making their resources to exploit opportunities
in the market, or neutralize those established by competitors. To be
strategic, resources should be valuable, rare, inimitable and
non-substitutable.

In \citet[177-180]{willcocks2003}, four types of sourcing options for
developing IT projects are outlined, of which three involve consultants.

\begin{enumerate}
\def\labelenumi{\arabic{enumi}.}
\tightlist
\item
  Internal development: has the the advantage of internalization of the
  learning outcomes, but comes with high costs related to mistakes and
  being the first mover.
\item
  Outsourcing: has the advantage of tapping into existing knowledge and
  experience, and the ability to get quickly up to speed. However,
  internalization of learning outcomes is not guaranteed, and
  consultants may not be familiar with existing organizational
  processes. For example, the development of an internal application by
  an external party.
\item
  Insourcing/partnering: has the same advantages as outsourcing, with
  the added bonus of facilitating the internalization of the learning
  outcomes. The disadvantage is mostly related to a more complex project
  management, with a variety of parties involved. For example: long-term
  contracts with IT consultants who operate side-by-side with an
  organization's own staff.
\item
  Cheap-sourcing: when IT projects are low risk, and far from the core
  business, and organization should consider cheap-sourcing. This option
  involves low investments and effort, but also comes with no internal
  learning. For example: development of a new promotional website by a
  digital agency.
\end{enumerate}

In the same research paper, \citet[188-189]{willcocks2003} identify two
congruent four-quadrant matrices to assess sourcing options.

\begin{itemize}
\tightlist
\item
  By business activity: non-critical, commoditized applications should
  be out-sourced. Critical, commoditized applications should be
  insourced or built in-house, and differentiating, critical
  applications should be built in-house or acquired.
\item
  By market comparison: A high-cost, low-quality market leads to
  in-house development, while a high-cost, high-quality market should
  lead to insourcing. A low-cost, low-quality market leads to
  cheap-sourcing and a low-cost, high-quality market is perfect for
  outsourcing.
\end{itemize}

By referring to commoditization, \citet{willcocks2003} implicitly refers
to asset specificity, blending elements of transaction cost economics in
the resource based view. \citet{watjatrakul2005} does this explicitly by
juxtaposing the resource-based view with the transaction cost view. Four
types of assets result from this exercise:

\begin{enumerate}
\def\labelenumi{\arabic{enumi}.}
\tightlist
\item
  Low specificity, non-strategic such as generic managerial
  capabilities.
\item
  Low specificity, strategic such as a configuration of capabilities
  that result in certain strategic decisions.
\item
  High specificity, non-strategic such as an consumer tracking
  technology that provides valuable insights into the organization's
  processes.
\item
  High specificity, strategic such as company experts that are
  responsible for developing a organization's differentiating features.
\end{enumerate}

\hypertarget{identification-theory}{%
\subsection{Identification Theory}\label{identification-theory}}

The research by \citet[311-313]{schwarz2005} claims that it matters
\emph{who} implements an IT project: ``technology-enabled inertia can be
explained through understanding an employee's social identifications and
his or her associated cognitions, where inertia exists on a sliding
scale of change.'' By defending their self-image, low-status groups can
hinder the implementation of an application. The sourcing assessment
needs to incorporate this finding.

``\ldots competitive pressures, on both clients and consultants,
combined with a US culture of anti-intellectualism and `macho' (grand
and unreflective) visions lead to the marketing and adoption of
simplistic and necessarily flawed techniques.'' \citep[ 34]{sturdy1998}

``Terms such as `growth' and `effectiveness' have mythical qualities.
They are condensation symbols collapsing a managerial world view into a
single word. So, too, {[}\ldots{]} consultancy packages {[}make{]} use
of condensation symbols thereby creating affective bonds to the symbol's
object, tying managers into the package at an emotional level and
creating a shared managerial language.'' \citep[ 290]{gill1993}

``Management consultants, their ideas, and their techniques play a
central role in creating the organization in such a way that it is
possible to control, change, and `improve' -- and at the same time,
reinforcing a positive managerial identity. While supporting managers in
handling their anxieties, some commentators have argued that by
reinforcing these, consultants create their own market.'' \citep[
48]{werr2002}

\hypertarget{embeddedness-theory}{%
\subsection{Embeddedness Theory}\label{embeddedness-theory}}

Embeddedness theorists distance themselves from transaction costs
economics \citep[ 14-16]{armbruster2006}. They argue that outsourcing
decisions are the byproduct of the relationships between the
decision-makers across different companies. Although transaction cost
economists stress the importance of trust within relational contracting,
it is undersocialized according to embeddedness theorists. ``As a
result, transactions may be inefficient without the participants either
noticing or calculating it as such. A transaction cost analysis of such
processes may then represent an ex post rationalization of an otherwise
inefficient solution.'' \citep[ 15]{armbruster2006}

Several empirical findings support embeddedness theory.

\hypertarget{sociological-neoinstitutionalism}{%
\subsection{Sociological
neoinstitutionalism}\label{sociological-neoinstitutionalism}}

A theory that is systematically drawn upon \citep[ 6-8]{armbruster2006}
is sociological neoinstitutionalism. It is based on the argument that
the belief in efficiency of certain practices or solutions drives
economic actions, rather than the proven efficiency. The result is that
large consultancies have been described as carriers, not only of
knowledge, but of legitimacy too. After all, it's their analyses that
validate management decisions.

\citet[20-21]{zucker1985} argues that this is the result of
trust-building signals (infra) growing beyond their initial goal of
delineating specific expectations. Trust-producing firms (such as
consultancy firms) an sich assume a high status, with the business world
protecting them against failure.

Clearly, this view is mainly appropriated by the critical view since it
raises doubts about the efficient outcomes on the practice of
consultancy.

\hypertarget{signaling-theory}{%
\subsection{Signaling Theory}\label{signaling-theory}}

Another theory that falls in the camp of the critical view is that of
economic signaling theory \citep[ 8-10]{armbruster2006}. Unlike
sociological institutionalism, it treats the economic actors as
experienced and knowledgeable, and not as part of an institution.
Signaling theory argues that in uncertain markets, suppliers invest in
features that signal status, quality and reliability.

\citet[15-16]{zucker1985} argues that these features can also be used to
signal similarity. While nationality, ethnicity and sex can indicate a
common cultural system, or a ``world held in common'', more superficial
(bought or acquired) features can delineate specific expectations in
specific situations. In consulting this translates into degrees,
certificates, using the adequate buzzwords, wearing a suit and driving a
quality car. These indicators signal adherence to the ``rules of the
game.''

Sociological neoinstitutionalism argues that legitimacy-seeking behavior
leads to inefficient market outcomes, while signaling theory argues the
opposite.

\hypertarget{problem-statement}{%
\section{Problem statement}\label{problem-statement}}

Organizations employing the services of a contingent workforce, like
digital consultants, should always be aware of principal-agent problems
arising from information asymmetries. This is caused by the fact that a
transaction between a vendor and a buying organization involves the
delivery of services in the future. As a result, opportunism is always
lurking around the corner.

Two types of opportunism can be identified \citet[242]{clark1993}:

\begin{enumerate}
\def\labelenumi{\arabic{enumi}.}
\tightlist
\item
  Ex ante: Pre-contractual opportunism or adverse selection
\item
  Ex post: Post-contractual opportunism or moral hazard
\end{enumerate}

Hiring consultants for digital services deserves a dedicated scope,
because of the following five properties that make it very prone to
opportunism \citep[ 207]{leslie1995}.

\begin{enumerate}
\def\labelenumi{\arabic{enumi}.}
\tightlist
\item
  Information technology evolves rapidly. Consequently, it involves a
  high degree of uncertainty. (ex ante)
\item
  The underlying economics of IT changes rapidly, making it hard to
  evaluate the consultant's contribution. (ex post)
\item
  IT has penetrated all business functions. It is hard to isolate it
  from organizational functions. The question is rarely about
  ``outsourcing or not'', but more often about ``what tasks will be
  outsourced, and how''. The result is that the responsibilities of
  in-house staff and consultants is very intertwined. (ex post)
\item
  The cost of switching providers are often significant. For example:
  some consultants have exclusivity for implementing a certain solution.
  (ex post)
\item
  Clients might be very inexperienced with regards to IT (and
  outsourcing it). This puts them at a disadvantage for selecting and
  evaluating a consultant (ex ante, ex post).
\end{enumerate}

\citet[59]{aubert1996} also points to a problem of measurement within IT
outsourcing. Contracts often specify all kinds of measures: response
time, uptime, error logs, etc. Although they are linked to explicit
provisions (such as fines, penalties and contract termination), there
are two conditions for them to be effective: (1) observability and (2)
verifiability. The former implies that the client can observe the actual
performance of the agent, while the latter is about verifying
observations and providing evidence.

\hypertarget{adverse-selection}{%
\subsection{Adverse Selection}\label{adverse-selection}}

Adverse selection is associated with the client's inability to determine
the client's capabilities with regards to the assignment. This analogous
to Akerlof's ``Lemons problem'' \citeyearpar{akerlof1970}: due to the
information asymmetry, clients don't want to pay more than the average
price for consultants within a certain niche. While consultants of
below-average quality (``lemons'') benefit from this average price,
above-average consultants will not want to compete, and are crowded out.

\citep[ 69-75]{armbruster2006} outlines various reasons for quality
uncertainty and groups them in two categories.

Category 1: Formal institutional uncertainty

\begin{itemize}
\tightlist
\item
  Consulting is an unbounded profession.
\item
  Consulting is an unbounded industry.
\item
  Consulting has unbounded service lines and product standards.
\end{itemize}

Category 2: Transactional Uncertainty

\begin{itemize}
\tightlist
\item
  Confidentiality
\item
  Product intangibility
\item
  Interdependent cooperation
\end{itemize}

The economic barriers to entry \citep[ 463]{fee2004} in IT consultancy
(and consultancy in general) are few to none. Anyone with experience in
a specific field, sector or technology can wrap it as advice and sell it
to whoever wants to hear it. Furthermore, the author of this paper is
unaware of legal barriers to entry.

According to some, assessing the quality of consultants is impossible.
For example, according do \citet[40]{bloomfield1995}, ``there can be no
presumed separation between technical skills and political skills, and
no ranking between the two in terms of their importance for consultancy
practice and the development of IT in user organizations.'' Furthermore,
\citet[101-102]{bettencourt2002} states that for knowledge-intensive
business services (KIBS) to succeed, a lot depends on the client.
``Client co-production roles {[}\ldots{]} are emergent, multi-faceted,
and highly collaborative because clients themselves possess much of the
knowledge and competence that a KIBS firm needs to successfully deliver
its service solution.''

For evaluating (future) performance of consultants, one has to rely on
informal and relational criteria \citep[ 277]{wright2002}. According to
\citet[250]{clark1993}, ``the main trust-producing mechanism
{[}\ldots{]} is the `closed' social structure; a form of individual
trust. The formal, institutional-based, trust-producing mechanisms are
weak. It is the contractual guarantees, and the history of past
transactions underlying reputation, which overcome the potential effects
of adverse selection and moral hazard.''

Especially in a situation where past transactions are absent,
``management consultancies must convey in some way to their clients that
they have something valuable to offer. {[}\ldots{]} consultants are able
to take control of the process by which impressions and perceptions of
their service are created. By managing the creation of these images
consultants are able to persuade clients of their value and quality.
Management consultancies are therefore `systems of persuasion' \emph{par
excellence} and impression management is not external to the core of
their work but is at its core.'' \citep[ 35]{clark1998}

In the existing literature, these remarks are part of a critical
paradigm regarding consultants \citep[ 4-5]{armbruster2006}. Authors
point to the contestable nature of consulting, the self-interest of
consultancy firms, and the stretching of consultancy advice.

See also:

\begin{itemize}
\tightlist
\item
  \citep{wright2002}
\item
  \citep{david2013}
\item
  \citep{mahoney2016}
\end{itemize}

Why does digital consultancy require a separate study? - Due to
long-term involvement in huge projects, can take long for reputational
info to spread and kick in - Vendor is tech-savvy \& can manipulate
online reputation

\hypertarget{moral-hazard}{%
\subsection{Moral Hazard}\label{moral-hazard}}

See \citep[ 72-73]{armbruster2006}.

Why does digital consultancy require a separate study? - Work from home
- Complexity lends itself to hiding fuckups. - Digital evolves quickly

\begin{itemize}
\tightlist
\item
  Vergelijking maken met andere soorten consultancies.
\end{itemize}

\hypertarget{governance-measures}{%
\subsection{Governance Measures}\label{governance-measures}}

In the existing literature, several governance mechanisms have been
proposed. What they all share is that they result in a higher degree of
trust\footnote{Trust is defined in the broad sense: ``the willingness of
  a party to be vulnerable to the actions of another party based on the
  expectation that the other will perform a particular action important
  to the trustor, irrespective of the ability to monitor or control that
  other party.'' \citep{kee1970} If the level of a perceived risk is
  bigger than the level of trust, the trustor will not engage in a
  risk-taking relationship.} between the business partners. More
specific, trust from the principal towards the agent.

\citet[265]{liberatore2010} found that building trust, and goal
congruence, can help solve the agency problem because it drives the
consultancy firm's motivation for short-term profits towards long-term
business and reputation.

\citet[193-194]{kirilov2012} breaks trust factors down in trust-building
and trust-sustaining factors. The former is about signaling ability
through references, experience and reputation. The latter is about
signaling integrity: effective and transparent communication,
proactivity, monitoring and consistently meeting contractual
obligations. This is highly compatible with the research by
\citet[717-720]{mayer1995}, which discovered three groups of trust
antecedents: ability, integrity and benevolence.

\begin{itemize}
\tightlist
\item
  \emph{Ability} describes the skills, competencies and characteristics
  that enable a party to have influence within a specific domain.
\item
  \emph{Integrity} involves the trustor's perception that the trustee
  adheres to the principles that the trustor finds acceptable for that
  domain.
\item
  \emph{Benevolence} is the extent to which the trustor believes that
  the trustee wants to do good to the trustor, and looks beyond their
  profit motive.
\end{itemize}

Below, a fairly exhaustive list of proposed governance measures is
outlined and grouped by their corresponding trust antecedent.

See also \citet{lewicki2006} and \citet{kirilov2012}.

\hypertarget{ability}{%
\subsubsection{Ability}\label{ability}}

\hypertarget{reputation}{%
\paragraph{Reputation}\label{reputation}}

\citet[193]{kirilov2012} describes reputation as ``a mixture of the
brand name of the enterprise, executive management background, maturity
level, customer references, and independent quality assessments.''

\citet[75-76]{armbruster2006} distinguishes three types of reputation:

\begin{enumerate}
\def\labelenumi{\arabic{enumi}.}
\tightlist
\item
  Public reputation is the perception of a consulting firm's (or
  individual consultant) past performance. While there are few to no
  barriers to enter the market as a whole with a newly-found consulting
  firm, public reputation is a huge barrier to reaching its upper end.
  Public reputation is like a public good ; the information is
  non-excludable and non-rivalrous.
\item
  Experience-based trust relates to personal experience with a specific
  partner. A positive relation drives future action. However, trust
  evolves slowly, and maintaining it requires commitment. That's why it
  is often constrained to a small group of business partners.
\item
  Networked reputation is the result of word-of-mouth recommendations
\end{enumerate}

\citet[243-244]{clark1993} asked 55 respondents about the factors that
are important when choosing an executive search \& selection
consultancy. ``Reputation of individual consultants'' and ``reputation
of the consultancy'' are in the top three factors. This reputation often
arises from ``a history of past transactions with individual
consultants. Frequent transactions between consultants and clients leads
to familiarity which underpins the latters' assessment of the former.''
In other words, because finding a new consultants implies a search cost
\citep[ 1072]{wilson2012}, incumbent consultants are expected to receive
new contracts as long as the cost incurred from a potential sub-optimal
performance is lower than the search cost of finding a new consultant.

The findings in \citet{clark1993} are confirmed by
\citet[285]{richter2009} who found, with regards to projects that
involve client-specific information, that ``clients are willing to
involve external consultants with whom they have established a
relationship of trust in the execution of such projects. {[}\ldots{]}
intermediate forms of governance between the extremes of market
procurement from an anonymous provider and fully-fledged integration,
are not only viable, but an effective option for clients to procure
managerial services.'' Organizations are more inclined to work with
consultants with whom they have no experience when no client-specific
information or industry expertise is required.

According to \citet[516]{nayyar1990} ``reputation performs as an
implicit contract. It is enforced by the seller's concerns about future
demand for the service provided. {[}\ldots{]} reputation is likely to
exhibit characteristics of a public good. Once acquired, it can be user
over and over again in the context of other services or markets.''

\hypertarget{third-party-assessors}{%
\paragraph{Third-party Assessors}\label{third-party-assessors}}

\citet[57-62]{zucker1985} describes the rise of the ``social overhead
sector'' in the 20th century. This sector acts as an ``intermediary'' in
a variety of situations: stock brokers, real estate agents, banks, etc.
The same principle can be applied to consultants: assessment by a
third-party agency can prevent adverse selection.

See \citep[ 76-77]{armbruster2006},

\hypertarget{integrity}{%
\subsubsection{Integrity}\label{integrity}}

\hypertarget{monitoring}{%
\paragraph{Monitoring}\label{monitoring}}

``At the post-contractual stage, agency theory asserts that monitoring
the agent gathers information about the agent and helps reduce
opportunism. Monitoring places an uncomfortable social pressure on the
agent that increases compliance. It also increases the principal's
ability to detect the agent's opportunism and thus to appropriately
reward or sanction agent behavior. It reduces the agent's motivation to
justify a failed strategy, and promotes actions consistent with
shareholder goals.'' \citep[ 13]{basu2011}

Researchers have emphasized three broad ways in which consultants should
be monitored \citep[ 15]{basu2011}:

\begin{enumerate}
\def\labelenumi{\arabic{enumi}.}
\tightlist
\item
  When the consultancy firm gives their agreement to the specified
  deliverables and accompanying deadlines.
\item
  During the implementation of a project, the client verifies that the
  deliverables are being produces according to the original plan by
  thoroughly and regularly assessing reviews and written and oral
  progress reports. Meeting with consultants is essential to ensure that
  consultants share all relevant information in a timely manner.
\item
  During the implementation of a project, the client checks that the
  consultants do not sacrifice quality nor scope to meet deadlines.
  Furthermore, the originally committed staff should not be changed
  without approval.
\end{enumerate}

\hypertarget{contractual-obligations}{%
\paragraph{Contractual Obligations}\label{contractual-obligations}}

\citet{mcfarlan1995} claim that it is important to have flexibility in
an outsourcing contract, because the target state of a project might
change due to evolving technology and business environment.
Nevertheless\ldots{}

\citet[4]{lacity2012} describes that there is substantial evidence that
positive outsourcing outcomes are associated with:

\begin{itemize}
\tightlist
\item
  more detailed contracts with regards to scope, service levels,
  responsibilities and adaption to change;
\item
  shorter-term contracts;
\item
  high-value contracts.
\end{itemize}

More detailed contracts, when resulting in requirements uncertainty, is
an enabler of goal congruence and trust between the consultancy firm and
the client, which is found to result in a better project performance
\citep{liberatore264}.

\hypertarget{formalised-transactional-contracting-purchasing-regulation}{%
\paragraph{Formalised Transactional Contracting (Purchasing
Regulation)}\label{formalised-transactional-contracting-purchasing-regulation}}

See \citet[4-5]{sturdy2021}

\hypertarget{whistleblowing}{%
\paragraph{Whistleblowing}\label{whistleblowing}}

Also media.

\hypertarget{incentivization}{%
\paragraph{Incentivization}\label{incentivization}}

``Basing the agent's rewards and incentives on imperfect surrogates of
performance lads to moral hazard, but aligning the preferences of the
agent and the principal through an appropriate reward structure helps
curb the agent's opportunistic behavior.'' \citep[ 13-15]{basu2011}
Several actions are proposed to align incentives: link payment to
completion of the promised deliverables, sharing of cost savings or
overruns with the consultancy firm, incentives and penalties related to
timely completion of a project.

\citet[264-266]{liberatore2010} finds that a higher goal congruence
between the consultancy firm and the client is an enabler of project
performance.

\hypertarget{clan-mechanisms-closely-related-to-incentivization}{%
\paragraph{Clan mechanisms (closely related to
incentivization)}\label{clan-mechanisms-closely-related-to-incentivization}}

See \citet[62]{aubert1996} and \citet{ouchi1980}.

\hypertarget{hostage-taking-outcome-driven-contracts-and-contingent-fees}{%
\paragraph{Hostage-taking, outcome-driven contracts and contingent
fees}\label{hostage-taking-outcome-driven-contracts-and-contingent-fees}}

When there are no trust-related nor legal enforcement mechanisms,
parties can be discouraged from forming long-term relationships.
According to \citet[47-48]{werner1993}, ``hostages'' are used in
situations where rational behavior would lead to sub-optimal outcome, in
the Paretoian sense. In game theoretical terms, hostage-taking is used
to prevent defecting behavior.

In terms of the subject of this paper, ``hostages'' could come in the
form of contingent fees. See \citep[ 243]{clark1993}

See \citet{tosi1997}.

\hypertarget{psychological-contract-obligations}{%
\paragraph{Psychological contract
obligations}\label{psychological-contract-obligations}}

According to \citet[357]{ang2004}, the legal interpretation of an IT
outsourcing contract is too limited. Instead, they claim that the
construct of a \emph{psychological contract} is more appropriate for
analyzing the relationship between an IT service supplier and customer.
The strength of psychological contract theory is threefold:

\begin{enumerate}
\def\labelenumi{\arabic{enumi}.}
\tightlist
\item
  it focuses on mutual obligations;
\item
  the emphasis is on psychological obligations;
\item
  the emphasis is on the individual level--not on the organizations as
  parties of the contract.
\end{enumerate}

Consequently, the psychological contract not only comprises the legal
contract, but also the unwritten promises, interpersonal relations, and
the individual interpretations and perceptions. Since consultancy
contracts can become extremely complex (with project descriptions going
into the ten thousands of words), and the involved parties entangled in
multiple ways, these intangible aspects can gain prominence. The
research in \citet[369-70]{ang2004} outlines several psychological
contract obligations that positively impact the success of an outsourced
IT project.

\begin{itemize}
\tightlist
\item
  On the supplier side: (1) clear authority structures, (2) knowledge
  transfer by educating the customer, (3) building inter-organizational
  teams.
\item
  On the customer side: (1) clear specification of requirements, (2)
  prompt payment, and (3) project ownership and monitoring.
\end{itemize}

Closely related is the work by \citet[9-13]{willcockskern} that makes a
distinction between the contractual level and the cooperative level. The
contractual level is about payment for the exchange of services and the
transfer of assets, information \& consultants. The cooperative level
involves formal communication mechanisms; personal investments in time,
resources \& knowledge; mutual goals \& objectives and social bonds. The
atmosphere surrounding the former is heavily impacted by developments at
the latter. A respondent in \citet[9]{willcockskern} states that ``the
contract is a bit like a nuclear deterrent. You need one and you have
got to have a framework, but if you've got to use it you are probably in
trouble.''

\hypertarget{infobase}{%
\paragraph{Infobase}\label{infobase}}

???

\hypertarget{benevolence}{%
\subsubsection{Benevolence}\label{benevolence}}

???

\hypertarget{tbd}{%
\subsubsection{TBD}\label{tbd}}

\hypertarget{regulation}{%
\paragraph{Regulation}\label{regulation}}

\begin{itemize}
\tightlist
\item
  Three sources of regulation can be identified (\citet{clark1993}
  246-247).
\item
  See \citet[813-817]{muzio2011}.
\end{itemize}

\hypertarget{government-initiated-codes}{%
\paragraph{Government Initiated
Codes}\label{government-initiated-codes}}

See \citet[3-4]{sturdy2021}

\hypertarget{self-imposed-sectoral-codes}{%
\paragraph{Self-imposed Sectoral
Codes}\label{self-imposed-sectoral-codes}}

See \citet[4]{sturdy2021}

\hypertarget{specific-commitment}{%
\paragraph{Specific Commitment}\label{specific-commitment}}

See \citet[12]{sturdy2021}

\hypertarget{research-questions}{%
\section{Research Questions}\label{research-questions}}

Abstractie maken van zaken zoals cultuur, bedrijfsgrootte, etc. Link met
HR \& internaliseren van externen, externe kennis.

1a. Why do Belgian firms rely on digital consultants? 1b. What are
inhibitors \& enablers for success of digital consultants? 1c.
Definition of success

\begin{enumerate}
\def\labelenumi{\arabic{enumi}.}
\setcounter{enumi}{1}
\tightlist
\item
  Do Belgian firms see PA problems with digital consultants?
\item
  Which control mechanisms do Belgium firms have in place with regards
  to adverse selection and moral hazard of digital consultants?
\item
  Which control mechanisms positively impact success of engaging with a
  consultancy firm?
\end{enumerate}

Research questions for side projects:

\begin{enumerate}
\def\labelenumi{\arabic{enumi}.}
\tightlist
\item
  Why do people join a consultancy firm or become an independent
  consultant?
\item
  Do reputational effects exist on the individual consultant level or on
  the firm level?
\end{enumerate}

\bibliographystyle{agsm}
\bibliography{references.bib}


\end{document}
