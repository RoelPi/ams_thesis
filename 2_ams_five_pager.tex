% !TeX program = pdfLaTeX
\documentclass[12pt]{article}
\usepackage{amsmath}
\usepackage{graphicx,psfrag,epsf}
\usepackage{enumerate}
\usepackage{natbib}
\usepackage{textcomp}
\usepackage[hyphens]{url} % not crucial - just used below for the URL
\usepackage{hyperref}

%\pdfminorversion=4
% NOTE: To produce blinded version, replace "0" with "1" below.
\newcommand{\blind}{0}

% DON'T change margins - should be 1 inch all around.
\addtolength{\oddsidemargin}{-.5in}%
\addtolength{\evensidemargin}{-.5in}%
\addtolength{\textwidth}{1in}%
\addtolength{\textheight}{1.3in}%
\addtolength{\topmargin}{-.8in}%

%% load any required packages here



% tightlist command for lists without linebreak
\providecommand{\tightlist}{%
  \setlength{\itemsep}{0pt}\setlength{\parskip}{0pt}}




\begin{document}


\def\spacingset#1{\renewcommand{\baselinestretch}%
{#1}\small\normalsize} \spacingset{1}


%%%%%%%%%%%%%%%%%%%%%%%%%%%%%%%%%%%%%%%%%%%%%%%%%%%%%%%%%%%%%%%%%%%%%%%%%%%%%%

\if0\blind
{
  \title{\bf On enhancing knowledge transfer from a contingent
workforce}

  \author{
        Roel Peters \\
    Antwerp Management School\\
      }
  \maketitle
} \fi

\if1\blind
{
  \bigskip
  \bigskip
  \bigskip
  \begin{center}
    {\LARGE\bf On enhancing knowledge transfer from a contingent
workforce}
  \end{center}
  \medskip
} \fi

\bigskip
\begin{abstract}

\end{abstract}

\noindent%
{\it Keywords:} 
\vfill

\newpage
\spacingset{1.45} % DON'T change the spacing!

Traditionally, management consultancy came in the form of strategic
advice. However, over the past decades, the business world has witnessed
that organizations increasingly focus on their core business, while
outsourcing everything else to a third party or a ``contingent
workforce'': consultants and freelancers \citep{msp2022}. This practice
has even penetrated organizations' core value chain, a phenomenon known
as business process outsourcing or BPO \citep{shi1}. However, working
with a contingent workforce, entails considerable risk.

\hypertarget{relevancy}{%
\section{Relevancy}\label{relevancy}}

\begin{enumerate}
\def\labelenumi{\arabic{enumi}.}
\tightlist
\item
  Trend: external \textgreater{} internal
\item
  Price of consultants
\item
  Hot topic: \emph{when} to rely on consultants?
\item
  Productivity: is it worth it? If so, under what circumstances?
\item
  Job market \& macro-economic aspects
\end{enumerate}

\hypertarget{state-of-the-literature}{%
\section{State of the literature}\label{state-of-the-literature}}

\hypertarget{what-is-it-consultancy}{%
\subsection{What is IT consultancy?}\label{what-is-it-consultancy}}

Defining IT consultancy is no easy task. Over the past decades, their
possible roles and variety of responsibilities have expanded
drastically. \citet[20-25]{swanson2010} has described five different
ways how consultants can contribute to an organization's innovation
process through IT.

\begin{itemize}
\tightlist
\item
  Business strategy: IT consultancy can lead the organization to new
  pursuits and technologies they wouldn't have discovered themselves.
  Second, IT consultancy can frame the need for innovation in strategic
  terms, and they prepare and legitimize the need for change.
\item
  Technology assessment: IT consultancy can facilitate the comprehension
  of IT technologies and its alternatives.
\item
  Business process improvement: Innovations that involve IT usually come
  to fruition only after business processes have been revamped. Business
  process changes usually require an outside-in view and offer rich
  opportunities for consulting.
\item
  Systems integration: In many cases, introducing a new technology
  requires that it needs to be integrated with existing systems and
  users need to be onboarded. This type of IT consultancy usually
  requires coding skills, hands-on design and implementation expertise
\item
  Business support services: Finally, once the implementation is
  completed, it can take a while before the solution is entirely
  assimilated. IT consultants can provide complementary IT services such
  as support and maintenance until the technology is entirely embedded
  in the organization. The line between consulting and outsourcing is
  becoming increasingly blurred.
\end{itemize}

\hypertarget{problem-statement}{%
\section{Problem statement}\label{problem-statement}}

An important risk is the \textbf{lack of knowledge transfer} from the
consultant to the hiring organization, especially in the case of complex
IT implementations. Why? When complex digital ecosystems are rolled out
in an organization, several factors determine the success of the
outcome. Not only should IT \& business strategies be aligned, and
should the solution be thoroughly adopted by business departments.
Equally important is how the IT department succeeds in developing and
absorbing knowledge about the solution and how it is embedded in the
organization. A lack of knowledge transfer between the contingent
workforce and the principal can results in a lack of understanding of
the solution's capabilities and applications. Several \textbf{negative
externalities} can arise:

\begin{itemize}
\tightlist
\item
  inadequate support towards business users;
\item
  incomplete maintenance with second-degree externalities such as risks
  for availability and security;
\item
  over time, a capability overlap with other tools in the organization's
  tool stack can develop.
\end{itemize}

There are many \textbf{causes for knowledge transfer to (partially)
fail}:

\begin{itemize}
\tightlist
\item
  The consultant simply does not have the required knowledge.
\item
  The consultant tries to lock in their client by not transferring all
  the required knowledge.
\item
  The organization (or the manager, or the hiring department) does not
  explicitly expect to extract knowledge from the consultant, but simply
  considers them to be a ``contingent workforce'' in its minimalist
  interpretation.
\item
  There are no adequate procedures, rituals and tools in place within
  the organization for facilitating the knowledge transfer.
\end{itemize}

\hypertarget{the-existing-body-of-knowledge}{%
\section{The existing body of
knowledge}\label{the-existing-body-of-knowledge}}

\hypertarget{management-consultancy}{%
\subsection{Management Consultancy}\label{management-consultancy}}

There is already a fair amount of research with regards to management
consultancy (i.e.~in the narrow sense of the term, meaning ``strategic
business advice'') that describes why consultants exist
\citep{canback1998, sturdy2009} and how they operate
\citep{clark1998, bessant1995, whittle2006}. Central in this literature
is the diffusion and transferring of knowledge. \citet{canback1999}
summarizes it neatly by stating that ``external consultants have a wider
knowledge base than their internal counterparts, having worked with more
clients and in a wider range of industries. Having seen similar problems
before, the cost of leveraging this knowledge base is lower for external
consultants.'' Despite this body of literature on management
consultancy, research that focuses on IT-related consultancy is fairly
scarce \citep{bloomfield1995, nevo2007, swanson2010}.

\hypertarget{knowledge-transfer}{%
\subsection{Knowledge transfer}\label{knowledge-transfer}}

There is substantial research on knowledge management (as a
multidisciplinary discipline within the field of information science)
and knowledge transfer (as a broad topic within the discipline).
Furthermore, there seems to be some academic interest in knowledge
transfer in a principal-agent context \citep{ning2008, haines2003}, as
is the case with between an organization and their contingent workforce.
This research could be key in steering and narrowing the scope of the
research.

\hypertarget{research-questions}{%
\section{Research Questions}\label{research-questions}}

Novel research is feasible for drawing conclusions regarding the
\emph{raison d'être} of IT consultants with regards to knowledge
transfer between IT consultants and their principals.

Given these observations, I propose the following research questions.

\begin{enumerate}
\def\labelenumi{\arabic{enumi}.}
\item
  How successful is knowledge transfer between IT implementation
  consultants and internal employees at corporations with regards to
  adoption and implementation of IT solutions? (Methodology:
  Quantitative such as surveys)
\item
  What factors have a positive impact on knowledge transfer between IT
  implementation consultants and internal employees of corporations?
  (Methodology: Qualitative research such as deep interviews and focus
  groups)
\end{enumerate}

The most relevant outcome of this research could be a set of
recommendations, or a framework for maximizing knowledge transfer in the
described setting.

\bibliographystyle{agsm}
\bibliography{references.bib}


\end{document}
