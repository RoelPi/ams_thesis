% !TeX program = pdfLaTeX
\documentclass[12pt]{article}
\usepackage{amsmath}
\usepackage{graphicx,psfrag,epsf}
\usepackage{enumerate}
\usepackage{natbib}
\usepackage{textcomp}
\usepackage[hyphens]{url} % not crucial - just used below for the URL
\usepackage{hyperref}

%\pdfminorversion=4
% NOTE: To produce blinded version, replace "0" with "1" below.
\newcommand{\blind}{0}

% DON'T change margins - should be 1 inch all around.
\addtolength{\oddsidemargin}{-.5in}%
\addtolength{\evensidemargin}{-.5in}%
\addtolength{\textwidth}{1in}%
\addtolength{\textheight}{1.3in}%
\addtolength{\topmargin}{-.8in}%

%% load any required packages here



% tightlist command for lists without linebreak
\providecommand{\tightlist}{%
  \setlength{\itemsep}{0pt}\setlength{\parskip}{0pt}}

% From pandoc table feature
\usepackage{longtable,booktabs,array}
\usepackage{calc} % for calculating minipage widths
% Correct order of tables after \paragraph or \subparagraph
\usepackage{etoolbox}
\makeatletter
\patchcmd\longtable{\par}{\if@noskipsec\mbox{}\fi\par}{}{}
\makeatother
% Allow footnotes in longtable head/foot
\IfFileExists{footnotehyper.sty}{\usepackage{footnotehyper}}{\usepackage{footnote}}
\makesavenoteenv{longtable}



\begin{document}


\def\spacingset#1{\renewcommand{\baselinestretch}%
{#1}\small\normalsize} \spacingset{1}


%%%%%%%%%%%%%%%%%%%%%%%%%%%%%%%%%%%%%%%%%%%%%%%%%%%%%%%%%%%%%%%%%%%%%%%%%%%%%%

\if0\blind
{
  \title{\bf (working title) Control mechanisms for working with
consultants}

  \author{
        Roel Peters \\
    Antwerp Management School\\
      }
  \maketitle
} \fi

\if1\blind
{
  \bigskip
  \bigskip
  \bigskip
  \begin{center}
    {\LARGE\bf (working title) Control mechanisms for working with
consultants}
  \end{center}
  \medskip
} \fi

\bigskip
\begin{abstract}

\end{abstract}

\noindent%
{\it Keywords:} 
\vfill

\newpage
\spacingset{1.45} % DON'T change the spacing!

\hypertarget{introduction}{%
\section{Introduction}\label{introduction}}

\hypertarget{relevancy}{%
\section{Relevancy}\label{relevancy}}

\begin{enumerate}
\def\labelenumi{\arabic{enumi}.}
\tightlist
\item
  Trend: external \textgreater{} internal
\item
  Price of consultants
\item
  Hot topic: \emph{when} to rely on consultants?
\item
  Productivity: is it worth it? If so, under what circumstances?
\item
  Job market \& macro-economic aspects
\item
  Recent news in BE/NL
\end{enumerate}

\hypertarget{defining-digital-consultancy}{%
\section{Defining ``Digital
Consultancy''}\label{defining-digital-consultancy}}

``Digital consultancy'' is defined as consultancy in (either or both)
technological and organizational aspects of digital transformation. The
following two sections substantiate this definition by elaborating on
the two concepts that comprise this definition. First, six properties
that define ``consultancy'' are discussed, followed by an elaboration on
``digital transformation.''

\hypertarget{consultancy}{%
\subsection{Consultancy}\label{consultancy}}

Consultants, or management consultants, have been described through a
multitude of metaphors and nicknames: ``capitalism's commissars''
\citep[ 93]{thrift2005}, ``shadowy figures operating in the background
but exercising considerable influence'' \citep[ 31]{kipping2012},
``agents of a modern rationalistic and universalistic culture'' \citep[
190]{kipping2012}, ``institutionally approved agents'' \citep[
193]{kipping2012}, ``marketized experts'' \citep[ 265]{furusten2012},
``magical figures, shamans or witch doctors'' \citep[ 68]{fincham2002},
``puppet masters'' \citep[ 69]{fincham2002}, ``knowledge entrepreneurs
that promote emotionally charged, enthusiastic, and unreasoned
discourse'' \citep[ 37]{leicht2006}, ``supra-experts'' \citep[
94]{kieser2006}, ``improvising experts'' \citep[ 272]{furusten2009} and
simply ``The Big Con'' \citep{mazzucato2023}.

Their work makes use of a vague body of knowledge described as
``elusive'', ``fuzzy'', ``perishable'', ``indeterminate'', ``esoteric'',
``fluid'' and ``changeable'' \citep{muzio2011}. Consequently, for buyers
it is hard to know what they need and what they get, and for sellers of
consultancy it is hard to know what to offer \citep[ 266]{furusten2012}.
Furthermore, consultancy is marked by very low professionalization since
occupational entry is unprotected, the supply of labor is unregulated
and there is no formal accreditation. \citep[ 20]{fincham2006}
Practicing ``consultancy'' is the main criterion of membership with
competences and ``time spent in the industry'' as the main
differentiators.

These remarks are prima facie evidence that no consensus around the
definition exists, and what they do is extremely hard to describe.
\citet[24]{kipping2012} states that ``definitions of management
consultancy are problematic because the permeable boundaries of the
industry have resulted in significant shifts over time in the
composition of the industry. This means that what comprises consulting
work is dynamic, ever shifting, and contested as new firms enter the
industry and techniques deemed formerly appropriate, change. Although
the industry is characterized by periodic structural shifts, at its
heart it is an advisory activity built on the client--consultant
relationship. {[}\ldots{]} it is perhaps this chimeral ability to avoid
precise definition and to be able to constantly reinvent its core
services to meet ever changing understandings of the problems that beset
contemporary organizations, which partly underpins its growing economic
importance.''

The following definition of ``consultancy'' is used throughout this
paper.

\begin{quote}
\emph{Consultancy is a service offered by an external service provider.
Although the responsibilities of a consultant are highly contingent on
the client organization and consultancy can take many forms and require
a variety of expertise, its goals is to establish change in the
procedures, organizational structure or tools of a client organization.
Finally, the success of a consultancy engagement is often determined by
the interactivity between a consultancy firm and the client
organization.}
\end{quote}

The sections below unpack this definition and elaborate on the six
properties that it comprises.

\hypertarget{external}{%
\subsubsection{External}\label{external}}

\citet[138]{chowdhury2021} describes consultants as ``external advisors
to corporations, nonprofits, governments, and any other forms of
organizations.''

\hypertarget{change}{%
\subsubsection{Change}\label{change}}

Clearly, the construct of a ``consultant'' cannot be described by the
topic that they work on, nor their academic and professional background,
accreditation or membership. Instead, we should look at their goal(s):
\citet[1]{werr1986} implies that there is always a change process
between clients and consultants. This is confirmed by
\citet[12]{kipping2000} who states that ``management consultancies earn
money through changing current procedures in client organizations.''
Although this change is often described by consultants as a `tailored
solution', consultants provide a service, which is inherently intangible
(compared to the `solid' nature of products) and hard to evaluate
\citep[ 348]{fincham1999}, especially because an evaluation should not
only account for the content of the changes, but also in terms of the
competence development of the client, as a result of the change process
\citep[ 17]{werr1986}.

Although consultants' goal is to establish change in an organization,
their role is often symbolic. As external consultants, associated with
their ``quest for knowledge'' and their ``quest for excellence'', they
are \citep[ 9-13]{pellegrin2006} well-equipped for legitimizing hard
decisions, signaling importance and providing meaning.

Establishing change is what sets consultancy apart from temporary
staffing. ``A temp is generally not supposed to change the work
practices at the client organization. A consultant is often expected to
do just that, or at least to provide an alternative point of view.''
\citep[ 5]{furusten2000}

The aspect of change is also important for drawing boundaries between
consultancy and outsourcing. Consultants are often hired for defining a
problem and presenting a solution at the same time \citep[
272]{furusten2009}, while outsourcing simply delivers the solution. In
that sense, outsourcing is the practice of obtaining goods or services
from an external provider, as a substitute for sourcing it internally
\citep[ 2]{lacity2012}, for a contractually agreed monetary fee and
period of time \citep[ 20-21]{leimeister2010}. Or in the words of
\citet[374]{zhu2001}: ``the process of transferring the responsibility
for a specific business function from an employee group to a
non-employee group.'' While early IT outsourcing initiatives were rather
``total'' \citet{willcocks1995} in nature, outsourcing individual
business functions is now a more common activity \citep[ 377]{zhu2001},
and focus has shifted from cost saving to quality, productivity,
flexibility and technological diversity \citep[ 185]{kirilov2012}. This
implies that outsourcing, unlike consultancy, is not about changing a
procedure or service, but rather ensuring continuation, or
implementation, by a third-party provider\footnote{\emph{Prima facie},
  this could be interpreted as a departure from the work by
  \citet{loh1992} and \citet{venkatraman1994}, who claim that IT
  outsourcing is an ``administrative innovation, {[}\ldots{]} involving
  significant changes in the routines used by the organization to deal
  with its tasks of internal arrangements and external alignments.''
  However, while there is indeed a firm-level change \citep[
  14]{nelson1985}, in the sense of who is responsible for a specific
  (part of a) procedure, the procedure or service involved does not
  change \emph{an sich.}}.

Finally, \citet[272-273]{furusten2009} argues that consultants not only
act as agents of change, but often as agents of stability. Consultants
are often perceived as ``relatively stable by {[}\ldots{]} employees,
stakeholders and other external counterparts'', this ``builds confidence
for the organization. The more stable an organization is experienced as
being, the greater its ability to concentrate on its core activities.''

\hypertarget{contingent}{%
\subsubsection{Contingent}\label{contingent}}

Being a consultant implies taking on a variety of responsibilities
throughout a certain time span. Although many consultants have
structured methodologies, which are converging across the industry
\citep[ 17]{werr1986}, ``{[}c{]}onsultants operate in an intense
environment that regularly entails new challenges.'' \citep[
138]{chowdhury2021} What entails consultancy work is dynamic, ever
shifting and contested with every new firm entering the industry and new
methodologies claiming the spotlight \citep[ 24]{kipping2012}.

This is what sets external consultancy apart from internal consultancy.

\hypertarget{relational}{%
\subsubsection{Relational}\label{relational}}

The nature of consultancy is often relational. First of all, a
consultant's work is embedded in an organization's web of interpersonal
relations. ``{[}T{]}he context, terms of reference, and ensueing
recommendations pertaining to a consultancy engagement may represent a
continuation, by other means, of ongoing processes of co-operation,
struggle and conflict between organizational groups.''
\citep{bloomfield1995}

Furthermore, a consultant's deliverable is often co-created with the
client. \citet[290-297]{nikolova2009} describes the client-consultant
relationship within a `social learning model'. It starts from the belief
that there is no ``knowledge out there'', and client and consultant need
to work closely together to develop problem solutions. The role of the
consultant is that of a ``facilitator of diagnosis and problem-solving''
and coach, while the client is the actual problem solver.
\citet[22]{clark1998} even goes so far as to claim that ``Like a bottle
of wine, a restaurant meal, or a book, the quality of a management
consultancy service is determined during enactment/consumption. This
indicates that the outcome of a consultancy service is highly dependent
upon the quality of the interaction between the client and the
consultant.''

A consultant needs to develop skills in order to fit in and adapt their
skill set to the needs of a client. When a relationship does not
succeed, no authority is placed on the skills of the service provider
\citep[ 10]{furusten2000} and the assignment might turnout to be
unsuccessful. In other words, a consultant is good at improvising:
``when they do not really know what problem the client has or how to
solve it, to improvise and act in a manner they believe the other party
expects of them in a particular situation is a convenient way for both
parties to muddle through.'' \citep[ 270]{furusten2009}

The importance of the relational aspect again sets consultancy apart
from outsourcing \citep[ 171-173]{kipping2012}. The outcome of a
consultancy assignment often depends heavily on the interaction between
the client and the service provider. On the other hand, outsourcing
focuses more on technical capabilities and implies an integral handover
of a (set of) service(s) to an external provider that becomes the sole
responsible for delivering them. Nevertheless, one could argue that the
implementation phase of a consultancy engagement offers the same
benefits as outsourcing certain IT functions. For this reason, where
appropriate, this study also uses insights from literature that focuses
on IT outsourcing.

Finally, the relational aspects also sets consultancy apart from
auditing, for which ``{[}i{]}ndependence is necessary to prevent
auditors from biasing their opinions in favor of their clients.''
\citep[ 310]{bazerman2011} In other words, to prevent client pressure
leading to client pleasing by the auditor \citep{koch2017}, audit
engagements shouldn't be relational, on the contrary.

\hypertarget{two-sided}{%
\subsubsection{Two-sided}\label{two-sided}}

Besides their tasks at the client, consultants also face internal
pressures from their employers in terms of optimized resource
utilization (billabillity), using proprietary knowledge and ``proximity
that they can develop with the client.'' \citet[138]{chowdhury2021} In
other words, a consultant serves two masters: the client organization,
and the organization that pays their wage. While the former expects an
adequate service, the latter has a commercial motive. \citep[
270]{furusten2012}

Two-sidedness is what sets external consultants apart from internal
consultants, but also from temporary staff. If a worker or contractor is
under the direct supervision of the organization it is working for, it
is seen as temporary staffing or ``staff augmentation.'' \citep[
1]{hodosi2019}

\hypertarget{diverse}{%
\subsubsection{Diverse}\label{diverse}}

Within this group, however, we can identify consultancy types: strategy
consulting, tax consulting, HR consulting, risk \& regulatory
consulting, etc. However, \citet[71-72]{armbruster2006} argues that the
boundaries between consultancy service types are blurred. A single
project often requires multiple types of services, but the distinction
is often artificial. Especially the boundary between strategy and IT
consultancy is opaque due to the fact that the big accounting \&
strategy firms entered the IT consultancy market to conduct
all-encompassing projects where strategy and IT meet.

\hypertarget{digital-transformation}{%
\subsection{Digital Transformation}\label{digital-transformation}}

\citet[28]{bloomfield1995} describes IT consultants as intermediaries:
``they interpose themselves between IT and clients, or between IT
suppliers and clients, in effect seeking to speak for technology. Put
another way, they seek to portray them selves as obligatory passage
points. {[}\ldots{]} the problem of choosing a particular functional
system is translated into a problem of choosing the best expert
advice.''

However, the offering by many different types of organizations of some
kind of IT consultancy for selecting, implementing, configuring and
preaching IT solutions has led to a blurring of consultancy work
\citetext{\citealp[ 31]{bloomfield1995}; \citealp[ 162]{kipping2012}}.
While consultants working for hardware and software vendors are claimed
to be motivated by the sale of their own products, consultancy firms aim
to equal themselves as suppliers of objective business advice.

These blurring lines are the result of the fact that there is more than
a technical dimension to IT solutions. Consequently, consultants frame
IT solutions not just from their technical dimension, but from their
organizational dimension, as well \citep[ 24-25]{bloomfield1995}. Like
strategy, technology, often depicted as neutral and separate from social
or political matters, can be wielded for political purposes. However,
the boundary between the merely technological and political is flexible:
a social or political problem can be translated as a technical one.

In accordance with this interpretation, the services of multinational
consultancy firms are defined or classified as consultancy in digital
operations (PwC); digital commerce \& engineering (Accenture); digital
transformation (EY, Bain, Deloitte, Tata); digital (McKinsey, KPMG);
digital, technology \& data (BCG); digitalization (Capgemini), digital
solutions (BoozAllenHamilton) and digital experience (Cognizant). Ergo,
it is remarkable that many research describes this group of consultants
as ``IT consultants''
\citep{nevo2007, loh1992, fincham2006, armbruster2006, bloomfield1995, schwarz2005},
while none of the big consultancy firms offer ``IT consultancy''.
\citet[96]{czerniawska1999} abandons the term and trades it in for
``IT-related consultancy''.

Mindful of these findings, this paper trades in the concept of ``IT
consultancy'' for ``digital consultancy'', and defines it as follows.

\begin{quote}
\emph{Digital consultancy is consultancy in (either or both)
technological and organizational aspects of digital transformation}.
\end{quote}

In this definition ``digital transformation'' refers to the expectation
that the use of digital technology will lead to favorable business
outcomes \citep[ 104-118]{wessel2020}, by redefining or supporting the
value proposition of an organization and imposing changes on the work
practices of its organizational members. Proceeding forward, this
interpretation rolls up the dichotomy between the concepts of ``digital
transformation'' and ``IT-enabled organizational transformation'' into a
single term.

\hypertarget{emergence-of-digital-consultancy}{%
\section{Emergence of Digital
Consultancy}\label{emergence-of-digital-consultancy}}

\hypertarget{the-pioneers-of-consultancy}{%
\subsection{The Pioneers of
Consultancy}\label{the-pioneers-of-consultancy}}

\hypertarget{the-ascent-of-consultancy}{%
\subsection{The Ascent of Consultancy}\label{the-ascent-of-consultancy}}

\hypertarget{the-macro-economic-context-supply-side-economics}{%
\subsubsection{The Macro-economic Context: Supply-Side
Economics}\label{the-macro-economic-context-supply-side-economics}}

The post-bureaucratic organization ``invites market dynamics into what
used to be intra-organizational matters and seeks to rid the
organization of activities that are not directly linked to its focal
service or product.'' \citep[ 3]{furusten2000}

Specifically for public sector use of consultants,
\citet[242]{ylonen2019} use the term `consultocracy'; the ``phenomenon
in which often short-term, outsourced expert knowledge production is
increasingly replacing the long-term work of civil servants and even
politicians. This results in an increased power of consultants over
politics, public governance, and public sector practices.''

\citet[250]{ylonen2019} argues that the increasing reliance on
consultants results in the loss of tacit knowledge in the public sector:
``The old bureaucratic virtues erode as informal trust and information
lose their organizational structures and channels.'' The culprit is
usually found in cost-cutting programs that generate pressure to
eliminate permanent work hours. However, it ``can also be advanced as a
hidden or explicit political agenda. {[}\ldots{]} major organizational
overhauls were often motivated by the desire to destruct existing
organizational structures. Constant change was desirable precisely
because it helped to shatter old ways of working.''

Finally \citet[252]{ylonen2019} describes how the increasing use of
consultancy weakens accountability of civil servants as responsibilities
are transferred to consultancy firms. Furthermore, in many countries,
contracts between the government and consultants are negotiated under
private law, putting severe limits on the public availability of
relevant documents.

\hypertarget{the-micro-economic-context-radical-reflexivity}{%
\subsubsection{The Micro-economic Context: Radical
Reflexivity}\label{the-micro-economic-context-radical-reflexivity}}

``Intrinsic to radical reflexivity is an `unsettling,' i.e., an
insecurity regarding the basic assumptions, discourse and practices used
in describing reality.'' \citep[ 370]{pollner1991}

A factor contributing to the emergence of ``professional service firms''
is the ``downsized, outsourced, subcontracted and global environment for
business services overall.'' \citep[ 17]{leicht2006}

``Markets may represent a new organizational logic, but markets do not
speak very clearly nor do they provide elite managers with clear
guidance regarding the directions markets are heading in specific
contexts. {[}\ldots{]} The institutional context is ripe for the rise of
management consulting as the profession that `interprets markets for
you.' The irony of this is that markets were supposed to obviate the
need for professional expertise. But what the recent changes in
institutional, societal logic have done is replace one set of ideologies
(professional institutions and practices as protectors of the public
good) with another one (markets as protectors of the public good). The
ability to institutionalize this professional role will depend on the
willingness of top corporate managers to pay for these types of services
and the ability of management consultants to prove that they can deliver
measurable results (higher profits; avoidance of legal and reputational
trouble, etc.).'' \citep[ 37]{leicht2006}

Some papers to explore:

\begin{itemize}
\tightlist
\item
  \citep[ 120-130]{armbruster2006}.
\item
  \citep{kipping2003}
\item
  \citep{kipping2012}
\item
  \citep[ 336]{fincham1999}
\item
  \citep{mckenna2006}
\end{itemize}

``Companies with effective in-house management or which operate in areas
of relative commercial stability may not need to employ consultancies.
Nevertheless, positive attitudes to consultancies and good practice in
their use appear particularly to be associated with rapidly changing
corporations. It appears that, within the context of growing
competition, the use of consultancies is a symptom of a much broader
sea-change in modern management approaches which underlies growing
instability in regional and local economies.'' \citep[ 662]{wood1996}

\hypertarget{digital-consultancy-a-logical-progression}{%
\subsubsection{Digital Consultancy: A Logical
Progression}\label{digital-consultancy-a-logical-progression}}

Three phases can be identified.

In the first phase, consultants were working with the first commercially
usable computers of the 1950s, as companies were exploring the benefits
of using them in their operations. IT consultancy emerged in three
groups of companies: technology companies, accounting and auditing
firms, and management consultancy firms. \citep[ 162]{kipping2012}

During the second phase, in the 1970s, managers began to employ IT to
align internal processes with their organizations' business objectives,
clearly pointing to an alignment of IT and strategy. The result was a
growing demand for IT consultants with a background in strategy.
Furthermore, as new application such as ERPs, CRMs, and accounting
software hit the market, IT became recognized as a facilitator of
change. IT consultants were working on client's operations and often
served as change managers.

The third phase arrived with the introduction of network computing, as
the internet fundamentally transformed the nature of commercial
transactions. This sparked the rise of a host of `dot.com consultancies'
that provides advice on how to exploit these new opportunities. This
third phase differs from the first two as it is mainly driven by new
applications and services, and not by new hardware developments. It is
during this phase that many IT consulting \& outsourcing services became
standardized, leading to rapid commoditization and lower prices. New
companies began to offer these services on a global scale, often driven
by a relatively cheap labor force in emerging economies.

``Where once a company would have spent considerable in-house time on
developing their own tailor-made software, most accept that they can buy
a ready-made package from a professional software house. Although many
companies are still in the grip of major software implementations, our
attitudes are changing. Ten years ago we believed that we should adapt
software to match our internal processes: nowadays we accept that it is
probably quicker, cheaper, and ultimately more effective to adapt our
processes to a given package.'' \citep[ 23]{czerniawska1999}

``It is already apparent that clients want their `advisers' to take a
much more hands-on role; strategy firms, for example, are constantly
exposed over their perceived reluctance to be involved in the
implementation of their recommendations.'' \citep[ 168]{czerniawska1999}

\hypertarget{why-digital-consultancy-practical-perspectives}{%
\section{Why digital consultancy: Practical
perspectives}\label{why-digital-consultancy-practical-perspectives}}

An end-to-end consultancy assignment involves many steps. The following
overview by \citet{turner1982} involves eight steps and demonstrates how
consultancy is external, contingent, relational and involves change.

\begin{enumerate}
\def\labelenumi{\arabic{enumi}.}
\tightlist
\item
  Provide requested information.
\item
  Provide solution to given problem.
\item
  Conduct diagnosis that may redefine problem.
\item
  Provide recommendations.
\item
  Assist with implementation.
\item
  Build consensus and commitment.
\item
  Facilitate client-learning.
\item
  Improve organizational effectiveness.
\end{enumerate}

The fifth goal (implementation) is not without controversy as
traditionally, some argued that ``one who helps put recommendations into
effect takes on the role of manager and thus exceeds consulting's
legitimate bounds.'' \citep{turner1982} Also, ``a frequent dilemma for
experienced consultants is whether they should recommend what they know
is right or what they know will be accepted.'' Finally, the author notes
that the last three steps are seen as by-products, and often not as
explicit goals.

If we go from consultancy, in general, to digital consultancy, it is
essential to keep the scope of the definition in mind. Digital
consultancy has a much broader scope than merely technological advice
and implementation. \citet{swanson2010} {[}20-25{]}\footnote{\citet{swanson2010}
  utilizes the term ``IT consultancy.'' However, by ascribing it to both
  IT-technical and organizational aspects of IT, it inherently refers to
  ``digital consultancy.''} has described five different ways how
consultants can contribute to an organization's innovation process
through information technology:

\begin{itemize}
\tightlist
\item
  \emph{Business strategy}: IT consultancy can lead the organization to
  new pursuits and technologies they wouldn't have discovered
  themselves. Second, IT consultancy can frame the need for innovation
  in strategic terms, and they prepare and legitimize the need for
  change.
\item
  \emph{Technology assessment}: IT consultancy can facilitate the
  comprehension of IT technologies and its alternatives.
\item
  \emph{Business process improvement}: Innovations that involve IT
  usually come to fruition only after business processes have been
  revamped. Business process changes usually require an outside-in view
  and offer rich opportunities for consulting.
\item
  \emph{Systems integration}: In many cases, introducing a new
  technology requires that it needs to be integrated with existing
  systems and users need to be onboarded. This type of IT consultancy
  usually requires coding skills, hands-on design and implementation
  expertise
\item
  \emph{Business support services}: Finally, once the implementation is
  completed, it can take a while before the solution is entirely
  assimilated. IT consultants can provide complementary IT services such
  as support and maintenance until the technology is entirely embedded
  in the organization.
\end{itemize}

See also \citep{bessant1995}.

In their 1994 study, for which they interviewed over 100 decision,
\citet[10-17]{lacity1994} group expectations with regards to outsourcing
into four categories: financial, business, technical and political
expectations.

\citet[656]{wood1996} discovered that ``consultancies tend to reinforce
the strategic strengths of experienced companies rather than compensate
for the weaknesses of the inexperienced.''

The following sections follow an amended classification by
\citet{lacity1994} and rely on research both in (IT) outsourcing and
consulting, since many conclusions apply to both. In these situations,
it is referred to as working with a ``third party.'' Nevertheless,
distinctions are mentioned wherever extrapolating conclusions from the
former to the latter is inadequate.

\hypertarget{financial-expectations}{%
\subsection{Financial Expectations}\label{financial-expectations}}

Financial expectations regarding digital consultancy and IT outsourcing
are twofold: reducing costs and improving financial control.

\hypertarget{cost-reduction}{%
\subsubsection{Cost Reduction}\label{cost-reduction}}

The expectation of cost reduction comes from the ability to save on
human resources, the ability to eliminate them in times of recession,
and not having to pay dues when a new technology needs to be explored
and adopted. \citet[10]{lacity1994} found that managers expect that
``unit costs are less expensive because of mass production efficiencies
and labor specialization.'' The former applies to outsourcing, but the
latter apply to both. For IT outsourcing, it also involves the
elimination of large fixed costs during recession and transferring
adjustment cost to a third-party \citep[ 52]{aubert1996}.

\hypertarget{cost-control}{%
\subsubsection{Cost Control}\label{cost-control}}

The second financial expectation is not necessary about reducing costs,
but about controlling them. \citet[233]{sturdy1998} describes how
``executives wanting to exercise control over the management and
investment of IT, but lacking the expertise.'' \citet[10]{lacity1994}
states that ``{[}i{]}n most organizations {[}\ldots{]} IT costs are
controlled through general allocation systems that motivate users to
excessively demand and consume resources.'' No surprise that involving
third parties is seen ``as a way to contain costs because vendors
implement cost controls that more directly tie usage to costs. In
addition, users can no longer call their favorite analysts to request
frivolous changes but instead must submit requests through a formal cost
control process.''

\citet[454]{ketler1993} found that some managers see outsourcing as a
means of sharing (financial) risks, reducing potential weaknesses in a
department. Problems (and associated costs) with staffing, technology
and selection are transferred to, or shared with the partner. However,
the author also states that other managers rather fear the risk of
quality loss, which hints at some kind of trade-off.

Finally, \citet{lacity1994} describes how some managers ``wanted to use
outsourcing to restructure their IT budgets from lumbering capital
budgets to more flexible operating budgets.''

\hypertarget{business-expectations}{%
\subsection{Business Expectations}\label{business-expectations}}

\hypertarget{developing-strategy}{%
\subsubsection{Developing Strategy}\label{developing-strategy}}

See \citet{sturdy1998}

\hypertarget{focusing-on-core-activities}{%
\subsubsection{Focusing on Core
Activities}\label{focusing-on-core-activities}}

\citet[272-273]{furusten2009} argues that consultants are often
perceived as ``relatively stable by {[}\ldots{]} employees, stakeholders
and other external counterparts'', this ``builds confidence for the
organization. The more stable an organization is experienced as being,
the greater its ability to concentrate on its core activities.''

\hypertarget{facilitating-mergers-acquisitions}{%
\subsubsection{Facilitating Mergers \&
Acquisitions}\label{facilitating-mergers-acquisitions}}

Oftentimes, ``{[}m{]}ergers and acquisitions create many nightmares for
IT managers, who are required to absorb acquired companies into existing
systems.'' \citep[ 12]{lacity1994} Although managers expect that
involving third parties for merging IT functions could solve technical
incompatibilities and absorb additional employees, the authors found
that this was rarely successful.

\hypertarget{technical-expectations}{%
\subsection{Technical Expectations}\label{technical-expectations}}

\hypertarget{hiring-skilled-individuals}{%
\subsubsection{Hiring Skilled
Individuals}\label{hiring-skilled-individuals}}

\citet[12]{lacity1994} describes that in many organizations, there is
dissatisfaction with in-house IT departments delivering systems late and
over budget. In that context, involving third parties is seen as a way
to improve technical service. \citet[233]{sturdy1998} agrees and states
that managers rely on consultants when their departments are ``lacking
the skills for a project'' or they want to have them ``compete with each
other.''

\citet[52]{aubert1996} found that as specialized firms have digital
services as their core business, which is a source of increased
efficiency and productivity. E.g. specialized firms can attract highly
skilled professionals which are in short supply in the market as a
whole.

The sought-after skills are not always technical in nature. More often,
it's about being up to date, and spotting emerging trends.
\citet[53]{werr2002} found that many organizations expect consultants
``to interpret the meaning of technological developments, industry
changes and emerging management concepts {[}\ldots{]} to the client
organization.''

\citet[452]{ketler1993} offers another interesting perspective: as the
scope of digital services expand, it is difficult or unnecessary for
small firms to have a (full-time) expert in each area. Specialized
third-party firms, on the other hand, offer a variety of skills and
technical knowledge to their clients.

\hypertarget{knowledge-transfer-diffusion}{%
\subsubsection{Knowledge Transfer \&
Diffusion}\label{knowledge-transfer-diffusion}}

In \citet[53]{werr2002}, a manager describes how they are often caught
up in day-to-day activities, and consultants can help them take a loot
at the ``big picture'', from a strategic perspective: they make sense of
the manager's organization in relation to its environment, such as its
competitors.

Return to \citet{turner1982}.

Something about knowledge transfer here \citep{sturdy2009}.

Nevertheless, there are constraints to knowledge transfer. According to
\citet[128-129]{cohen1990}, ``the ability to evaluate and utilize
outside knowledge is largely a function of the level of prior related
knowledge {[}such as{]} basic skills, or even a shared language but may
also include knowledge of the most recent scientific or technological
developments in a given field. {[}\ldots{]} These abilities collectively
constitute what we call a firm's \emph{absorptive capacity}.''

\citet[84]{fincham2002} made an interesting observation:
organization-specific knowledge and expert knowledge are very complexly
related, and knowledge transfer can only happen into a ``well-prepared
ground.'' \citet[922]{nooteboom2000} describes this from a transaction
cost economics perspective: ``{[}O{]}ne needs to make investments that
are `specific' to the relation, {[}\ldots{]} and a certain durability of
the relation is required to set up and recoup the investment.''
Especially tacit knowledge (impossible to codify or document) suffers
from this problem, as it can only be transferred through direct
interaction and with hands-on participation by the intended recipient.

``Managers thus viewed consultants as a way of bypassing the knowledge
filters created by the organizational hierarchy, as well as the effects
of organizational politicals, which became salient in times of
reorganization and change. Management consultants were seen as a way for
managers to gain a `true' picture of what was going on in their
organizations.'' \citep[ 54]{werr2002}

``House consultants also had accumulated a unique understanding of the
client company's historical legacies, having a much longer time
perspective than individual managers who frequently changed jobs. The
consultants were thus described as the `organizational memory' of the
organization.''

One of the possible roles of (management) consultants described by
\citet[269]{furusten2009} is that of the ``carrier'': ``Carriers of
experience, expertise, knowledge, information and data about leadership,
management, organization, top-down strategies and holistic
perspectives.''

\hypertarget{political-expectations}{%
\subsection{Political Expectations}\label{political-expectations}}

Finally, there are also reasons that are beyond the business-economical
sphere. Oftentimes, individuals want to, or need to, pursue their
personal goals within an organization. They might have ideas or plans,
and use ``the `objectivity' and/or status of consultants to legitimate
or influence a course of action.'' \citep[ 233]{sturdy1998}

Political expectations can also come from outside an organization. The
use of consultants can instil trust in shareholders and other
stakeholders. As \citet[70]{kieser2006} noted: ``Companies that are held
internally and externally accountable for how they `handle' uncertainty
will contract consultants as a sign of good management. Even patients
who principally distrust physicians cannot avoid consulting them.''

\citet[258]{lacity1993} also identify a number of political motives, in
which managers simply hired consultants to jump on the bandwagon: to
react to the efficiency imperative, justify additional resources, react
to positive outsourcing media reports or enhance credibility.

\citet[54]{werr2002} describes how consultants can be used as a leading
example. They are ``valued for energizing the change efforts and pushing
the change projects forward. {[}\ldots{]} In supporting the realization
of change projects, consultants provided methodology as well as an
`energizing example' with their own style of working.''

\hypertarget{why-digital-consultancy-theoretical-perspectives}{%
\section{Why digital Consultancy: Theoretical
perspectives}\label{why-digital-consultancy-theoretical-perspectives}}

According to \citet[3-6]{armbruster2006}, the theoretical perspectives
on consultancy can be broken down into two main categories and
corresponding streams of literature. The first one is the functionalist
view, which sees consultants as ``carriers and transmitters of
management knowledge.'' The second perspective argues that the
functionalist perspective is to narrow in scope to grasp consulting
projects: client-consultants interactions are open to distortions, and
understanding them requires research. This is known as the critical
view.

\hypertarget{transaction-cost-economics}{%
\subsection{Transaction Cost
Economics}\label{transaction-cost-economics}}

Transaction costs economics sees economic organization as a problem of
contracting, i.e.~organizing economic activity. The starting point is
that every transaction comes with certain costs, both ex ante and ex
post.

\citet[147-148]{dahlman1979} obtains a classification of transaction
costs by going through the different phases of the transaction process.

\begin{itemize}
\item
  Search \& information costs (ex ante): ``Two parties {[}\ldots{]}
  search each other out, which is costly in terms of time and resources.
  If the search is successful {[}they{]} must inform each other of the
  exchange opportunity {[}\ldots{]} and the conveying of such
  information will again require resources.''
\item
  Bargaining \& decision costs (ex ante): ``Often {[}\ldots{]} agreeable
  terms can only be determined after costly bargaining between the
  parties involved.''
\item
  Policing \& enforcement costs (ex post): ``After the trade has been
  decided on, there will be the costs of policing and monitoring the
  other party to see that his obligations are carried out as determined
  by the terms of the contract, and of enforcing the agreement
  reached.''
\end{itemize}

The last type of transaction costs arises from bounded rationality: it
is impossible to estimate both the costs and risks of complete
contracts, or even enacting and enforcing them. \citep[ 53]{aubert1996}
The result is that the contractual partners often decide to leave room
for adaptation and interpretation, which, in turn, increases the risk of
opportunistic behavior (infra).

What follows is that the decision whether a service should be purchased
in the market is the result of a comparison of the resulting costs
(including transaction costs) with the costs of producing within a
``hierarchy''. While hierarchies coordinate the flow of materials
through sequential steps by controlling it on a higher level in the
managerial hierarchy, markets coordinate them through the supply and
demand forces between firms \citep[ 485]{malone1987}.

Typically, as they acquire new resources, hierarchies can specialize,
resulting in a higher productivity. However, ``{[}a{]}s a firm gets
larger, there may be decreasing returns to the entrepreneur function,
that is, the costs of organi{[}z{]}ing additional transactions within
the firm may rise. {[}For example, because{]} the entrepreneur fails to
place the factors of production in the uses where their value is
greatest.'' \citep[ 394-395]{coase1937} The result is that firms
increasingly are exposed to costs of internal coordination:
``{[}E{]}very time that a job which was previously done by one man or
one group of men is divided into two or more parts, the problem of
coordinating the work of the now separated groups or individuals begins
to arise.'' \citep[ 40]{robinson1931}

In other words, as companies aim to reduce production costs, by
increasing scale, they specialize, albeit increasing coordination costs.
If coordination costs would not exist, organizations would grow
indefinitely. The result is that blue-collar jobs disappear as
production costs are reduced, while white-collar jobs, aimed at
coordination, do not. \citep[ 31-32]{canback1998} Mindful of this
evolution, the assumption is that there is a high demand for advice and
(IT) solutions that improve coordination within and between firms. These
are services in which consultants are particularly well-versed. The
question to ask here is: are the transaction costs for working with
external consultants lower than internal coordination costs when it
comes to improvement of internal coordination and knowledge production?

\citet[37-44]{canback1998} argues it does, and starts from the three
critical dimensions of transactions, a popular research topic within
transaction cost economics.

\emph{Asset specificity} describes the degree to which physical, human
or site assets have a specific usage and can not be put to use for
another purpose. With highly idiosyncratic transactions, market forces
fail as no vendor is willing to tailor his product or service to one
client, and face downward price pressure, since the latter acts as a
monopsonist \citep[ 218-228]{robinson1969}, and no buyer is willing to
put its faith in the hand of a third party at the risk of being
blackmailed. The result is a bilateral monopoly. \citep[
63]{williamson1985}

That's why, according to \citet[250-253]{williamson1979} higher asset
specificity leads either to one of two forms of ``relational
contracting''. The first form is bilateral governance in which there are
``admissible dimensions for adjustment such that flexibility is provided
under terms in which both parties have confidence.'' The second form is
unified governance (i.e.~internalization or vertical integration), in
which ``adaptations can be made in a sequential way without the need to
consult, complete, or revise inter-firm agreements. Where a single
ownership entity spans both sides of the transactions, a presumption of
joint profit maximization is warranted.''

Initially, \citet[95-96]{williamson1985} identifies 4 sources of asset
specificity, but two other sources have been added:

\begin{enumerate}
\def\labelenumi{\arabic{enumi}.}
\tightlist
\item
  \emph{site specificity}, or the degree to which the successive stages
  of production are in close proximity to each other;
\item
  \emph{physical asset specificity}\footnote{The service sector
    equivalent is also known as \emph{procedural asset specificity}, or
    specific routines and workflows tailored to a particular
    transactional relationship, which are hard to modify or redeploy
    without value reduction. \citep{zaheer1995}}, or the degree to which
  the physical properties of the product are unique;
\item
  \emph{human asset specificity}, or the degree to which the skills, or
  configuration of skills within a team, are unique to an organization's
  production process;
\item
  \emph{dedicated assets}, or general investments by the seller which
  are made with the expectation of a considerable amount of trade with
  one particular buyer;
\item
  \emph{brand capital specificity} \citep[ 335]{vita2011}, or investment
  in reputation that could be harmed by a vendor delivering bad quality;
\item
  \emph{temporal specificity} \citep[ 486]{malone1987}, or the degree to
  which an asset's value is dependent on it reaching the user within a
  certain time limit such as shipbuilding or hotel linen delivery.
\end{enumerate}

The second transaction dimension is its \emph{frequency}. A one-time
transaction with high asset specificity does not require a different
contracting approach, because there is no subsequent phase in which the
buyer/vendor can leverage his monopsony/monopoly power and stray from
the initial contract. However, when the frequency goes beyond a single
transaction ``idiosyncratic transactions are ones for which the
relationship between buyer and supplier is quickly thereafter
transformed into one of bilateral monopoly.'' \citep[
241]{williamson1985}

\emph{Uncertainty}. Within the context of transaction cost economics,
\citet[38]{shin2003} states that ``many empirical studies show mixed and
contradictory results against what transaction cost economics predicts,
especially for the concept of uncertainty.'' As a solution, he reduces
the concept to ``behavioral uncertainty'', ignoring environmental
uncertainty \citep[ 391-392]{watjatrakul2005}. This is in line with the
original interpretation by \citet[79]{williamson1985}: ``The proposed
match of governance structures with transactions considers only two of
the three dimensions for describing transactions: asset specificity and
frequency.'' Uncertainty arises from these two, and together with
bounded rationality and opportunism, gives rise to exchange difficulties
\citep[ 7]{williamson1975}, making it ``more imperative to organize
transactions within governance structures that have the capacity to
`work things out.'\,'' \citep[ 79]{williamson1985} In this context,
behavioral uncertainty can't be disentangled from asset specificity.
\citet[78]{williamson1985} is fully aware that this is a departure from
Coase's transaction cost rationale.

To drive back the theory to the subject of consulting,
\citet[37]{canback1998} argues that it's low mainly human asset
specificity that favors the use of consultants, referring to their
solutions and advice that can easily be reproduced at many
organizations.\footnote{Furthermore, Canback claims that transaction
  frequency and uncertainty are less of an influence. By referring to
  market uncertainty, not only does he obscure the fact that consultants
  rather thrive in a context with high complexity and uncertainty, he
  also misrepresents the uncertainty dimension that is central in
  transaction cost economics. This is a prime example of the vagueness
  surrounding the concept of uncertainty in transaction cost economics
  (supra).} \citet[408]{watjatrakul2005} put the theory to the test and
compared the transaction cost view with the resource-based view (infra)
for describing the sourcing decisions in three cases and comes to the
following conclusion: ``a high-specificity asset has a major impact on
sourcing decisions. It overpowers the effect of uncertainty.''

Focusing on low-specificity assets allows consultancy firms to achieve
economies of scale. That's why they rather shun highly idiosyncratic
assignments. Rather, they'll focus on (often high-level) organizational
advice and IT architectures, since these have the biggest adaptive
properties.

Borrowing rhetoric from the resource-based view (infra),
\citet[498]{mata1995} applies Canback's conclusion on technical IT
skills: ``While technical skills are essential in the use and
application of IT, they are usually not sources of sustained competitive
advantage. {[}\ldots{]} they are usually not heterogeneously distributed
across firms. Moreover, even when they are heterogeneously distributed
across firms, they are typically highly mobile. {[}\ldots{]} firms
without the required analysis, design, and programming skills required
to make an IT investment can hire technical consultants and
contractors.'' Ergo, digital capabilities are very likely to be
outsourced.

\citet[16-17]{nevo2007} also concludes that his research supports the
transaction cost hypothesis: ``when the internal IT capability is weak,
developing and implementing an IT solution is likely to cost more
compared with hiring external IT consultants to do the same
job.''\footnote{The reverse situation also supports the identification
  theory: ``IT consultants will not receive the legitimacy they require
  {[}\ldots{]} if their knowledge and expertise do not differ from that
  possessed by the in-house IT team. Under these circumstances, external
  IT consultants' impact on IT productivity is expected to be lower.''
  \citep[ 17]{nevo2007}}

\hypertarget{agency-theory}{%
\subsection{Agency Theory}\label{agency-theory}}

At the center of agency theory is the observation that the firm is not
an individual. \emph{Au contraire,} ``the \emph{behavior} of the firm is
like the behavior of a market; i.e., the outcome of a complex
equilibrium process.'' \citep[ 311]{jensen1976} When (the owners of) an
organization ask(s) its employees to deliver a service, or buys a
service in the market, it encounters the two-pronged principal-agent
problem\footnote{Agency theory is also concerned with the problem of
  risk sharing that arises when principal and agent each prefer
  different courses of action due to differing attitudes towards risk.
  \citep[ 58]{eisenhardt1989} However, this is not within scope of this
  paper.}. By delegating work to an agent, the principal has to account
for the fact that ``(a) the desires or goals of the principal and the
agent conflict and (b) it is difficult or expensive for the principal to
verify what the agent is doing.'' \citep[ 58]{eisenhardt1989} These
phenomena are known as goal incompatibility and information asymmetry.

\begin{itemize}
\tightlist
\item
  \emph{Goal incompatibility}: The organization is interested in a
  timely roll-out of a quality solution for a problem they have. The
  consultancy firm, on the other hand, is driven by profit maximization.
\item
  \emph{Information asymmetry}: consultants wield enormous power over
  the knowledge that they possess and use. Their clients depend on this
  knowledge, making them vulnerable, and putting them at mercy of the
  consultant. \citep{brien1998} For example, \citet[248-249]{ylonen2019}
  found that payments from the Finnish Ministry of Finance to a
  particular consultancy company drastically increased year-over-year,
  because the consultancy firm had a quasi-monopoly in the knowledge
  regarding particular remuneration practices. Furthermore, when a
  particular individual sells his services to another one, it may be
  difficult to asses its true value \citep[ 134-135]{ouchi1980}. This is
  especially the case when interdependent technologies are involved, as
  their implementation and maintenance requires teamwork. On that
  account, disentangling individual contribution from the team's joint
  efforts are particularly hard. This situation invites opportunism such
  as slacking off.
\end{itemize}

These phenomena could lead to \emph{adverse selection} on the one hand
and \emph{moral hazard} on the other hand. \citep[ 14]{rousseau1993} The
former refers to the contracting of agents ``unqualified to fulfill
their end of the bargain due to active misrepresentation of the agent's
competence and expertise.'' The latter occurs when ``the agent shirks
and reduces his or her efforts.''

To prevent adverse selection and/or moral hazard, control mechanisms
could be put into place, resulting in agency costs. These include the
costs of structuring, monitoring, and bonding a set of contracts among
agents with conflicting interests, plus the residual loss incurred
because the cost of full enforcement of contracts exceeds the
benefits.'' \citep[ 327]{fama1983} The following expenditures sum up to
the total amount of agency costs \citep[ 308]{jensen1976}:

\begin{enumerate}
\def\labelenumi{\arabic{enumi}.}
\tightlist
\item
  monitoring expenditures;
\item
  bonding expenditures to achieve incentive alignment;
\item
  residual loss: the remaining ``loss of welfare'' of the principal in
  those situations where the agent makes divergent decisions because it
  was to expensive to offset it through bonding or monitoring.
\end{enumerate}

Circling back to the this paper's subject of digital consulting, we can
summarize the interplay of transaction cost economics and agency theory
as follows. Low asset specificity of knowledge production and the
improvement of internal coordination mechanisms through IT leads firms
to order these services in the market. However, human bounded
rationality does not allow for detailed consultancy contracts, since it
is impossible for humans to predict the future. Payment for consulting
engagements are typically recurring (high frequency), which leads to
behavioral uncertainty, since the incentives of the consulting firm and
the client do not align (goal incompatibility), their complex
interdependent services are hard to evaluate and their contributions are
hard to measure (information asymmetry). This could lead to adverse
selection and moral hazard, and asks for control mechanisms to economize
on agency costs.

Finally, an extra remark is in place. One should not forget that
managers are also agents themselves. As \citet[584]{tosi1997} describes:
``the reality {[}is{]} that in large organizations, owners may be
separated from the managers who make decisions in forms, and that the
two may have different interests.'' In that sense, \citet{fincham2002}
claims that a consultant can be described as ``an agent's agent'',
extending the management's own agency function. In other words,
consultants operate at the ``outer reaches of corporate power'',
stretching corporate authority to its limits. Consequently, their
legitimacy is often problematic within the corporation that engages with
them.

\hypertarget{resource-based-view}{%
\subsection{Resource-based View}\label{resource-based-view}}

The resource-based view rejects the traditional microeconomic
assumptions that goods or services are homogeneous. Instead, it argues
that they are heterogeneously distributed across firms, and may be long
lasting due to not being perfectly transferable, for example because of
resource immobility \citetext{\citealp[ 392]{watjatrakul2005}; \citealp[
491]{mata1995}}.

These resources come in the form of assets, capabilities or
organizational processes. Firms can obtain above-normal results if they
can establish a competitive advantage by making their resources exploit
opportunities in the market, or neutralize those established by
competitors. To be strategic, resources should be valuable, rare,
inimitable and non-substitutable.

\citet[495-500]{mata1995} identifies five attributes of IT as sources of
sustained competitive advantage for a firm.

\begin{enumerate}
\def\labelenumi{\arabic{enumi}.}
\tightlist
\item
  Customer switching costs: the ability to create a lock-in effect for
  one's customers through the use of IT.
\item
  Access to capital: IT (or digital) investments can be very expensive
  and risky. Only a couple of firms might be able to acquire the capital
  to make these investments.
\item
  Proprietary technology: when technology can be kept proprietary, it
  can be a source of sustained competitive advantage.
\item
  Technical IT skills: the ability to attract and keep technical IT
  skills that is required to built IT applications.
\item
  Managerial IT skills: management's ability to conceive of, develop and
  exploit IT applications to support and enhance other business
  functions can enable a firm to manage market risks associated with IT
  investments.
\end{enumerate}

In \citet[177-180]{willcocks2003}, four types of sourcing options for
developing IT projects are outlined, of which three involve consultants.

\begin{enumerate}
\def\labelenumi{\arabic{enumi}.}
\tightlist
\item
  Internal development: has the the advantage of internalization of the
  learning outcomes, but comes with high costs related to mistakes and
  being the first mover.
\item
  Outsourcing: has the advantage of tapping into existing knowledge and
  experience, and the ability to get quickly up to speed. However,
  internalization of learning outcomes is not guaranteed, and
  consultants may not be familiar with existing organizational
  processes. For example, the development of an internal application by
  an external party.
\item
  Insourcing/partnering: has the same advantages as outsourcing, with
  the added bonus of facilitating the internalization of the learning
  outcomes. The disadvantage is mostly related to a more complex project
  management, with a variety of parties involved. For example: long-term
  contracts with IT consultants who operate side-by-side with an
  organization's own staff.
\item
  Cheap-sourcing: when IT projects are low risk, and far from the core
  business, and organization should consider cheap-sourcing. This option
  involves low investments and effort, but also comes with no internal
  learning. For example: development of a new promotional website by a
  digital agency.
\end{enumerate}

In the same research paper, \citet[188-189]{willcocks2003} identify two
congruent four-quadrant matrices to assess sourcing options.

\begin{itemize}
\tightlist
\item
  By business activity: non-critical, commoditized applications should
  be out-sourced. Critical, commoditized applications should be
  insourced or built in-house, and differentiating, critical
  applications should be built in-house or acquired.
\item
  By market comparison: A high-cost, low-quality market leads to
  in-house development, while a high-cost, high-quality market should
  lead to insourcing. A low-cost, low-quality market leads to
  cheap-sourcing and a low-cost, high-quality market is perfect for
  outsourcing.
\end{itemize}

By referring to commoditization, \citet{willcocks2003} implicitly refers
to asset specificity, blending elements of transaction cost economics in
the resource based view. \citet{watjatrakul2005} does this explicitly by
juxtaposing the resource-based view with the transaction cost view. Four
types of assets result from this exercise:

\begin{enumerate}
\def\labelenumi{\arabic{enumi}.}
\tightlist
\item
  Low specificity, non-strategic such as generic managerial
  capabilities.
\item
  Low specificity, strategic such as a configuration of capabilities
  that result in certain strategic decisions.
\item
  High specificity, non-strategic such as an consumer tracking
  technology that provides valuable insights into the organization's
  processes.
\item
  High specificity, strategic such as company experts that are
  responsible for developing a organization's differentiating features.
\end{enumerate}

\hypertarget{identification-theory}{%
\subsection{Identification Theory}\label{identification-theory}}

The research by \citet[311-313]{schwarz2005} claims that it matters
\emph{who} implements an IT project: ``technology-enabled inertia can be
explained through understanding an employee's social identifications and
his or her associated cognitions, where inertia exists on a sliding
scale of change.'' By defending their self-image, low-status groups can
hinder the implementation of an application. The sourcing assessment
needs to incorporate this finding.

``\ldots competitive pressures, on both clients and consultants,
combined with a US culture of anti-intellectualism and `macho' (grand
and unreflective) visions lead to the marketing and adoption of
simplistic and necessarily flawed techniques.'' \citep[ 34]{sturdy1998}

``Terms such as `growth' and `effectiveness' have mythical qualities.
They are condensation symbols collapsing a managerial world view into a
single word. So, too, {[}\ldots{]} consultancy packages {[}make{]} use
of condensation symbols thereby creating affective bonds to the symbol's
object, tying managers into the package at an emotional level and
creating a shared managerial language.'' \citep[ 290]{gill1993}

``Management consultants, their ideas, and their techniques play a
central role in creating the organization in such a way that it is
possible to control, change, and `improve' -- and at the same time,
reinforcing a positive managerial identity. While supporting managers in
handling their anxieties, some commentators have argued that by
reinforcing these, consultants create their own market.'' \citep[
48]{werr2002}

``management consulting is the new and relatively recent attempt to take
advantage of the destructured business environment of corporate clients
that purchase business services. They represent a semi-institutionalized
attempt to advance the professional aspirations of managers themselves,
especially in light of the well-publicized attacks on middle management
infrastructure that has accompanied the latest corporate downsizing
waves.'' \citep[ 35]{leicht2006}

``Part of the lure of management consulting lies with the ability to
work with highly prestigious business clients on important business
problems in an environment well oiled by high fees and salaries.''
\citep[ 37]{leicht2006}

\hypertarget{embeddedness-theory}{%
\subsection{Embeddedness Theory}\label{embeddedness-theory}}

Embeddedness theorists distance themselves from transaction costs
economics \citep[ 14-16]{armbruster2006}. They argue that outsourcing
decisions are the byproduct of the relationships between the
decision-makers across different companies. Although transaction cost
economists stress the importance of trust within relational contracting,
it is undersocialized according to embeddedness theorists. ``As a
result, transactions may be inefficient without the participants either
noticing or calculating it as such. A transaction cost analysis of such
processes may then represent an ex post rationalization of an otherwise
inefficient solution.'' \citep[ 15]{armbruster2006}

\citet[992]{nooteboom1996} states that typically, economists tend to
neglect intrinsic utility and that it doesn't matter who the transaction
partner is. Embeddedness theory rejects this, as personality and social
embeddedness enter the picture.

Embeddedness solves the adverse selection problem (infra) since
``{[}p{]}rincipals frequently know their agent's type because of
personal familiarity with potential agents or through members of trusted
social networks in which both principal and agent are embedded.''
\citep[ 277]{shapiro2005}

Several empirical findings support embeddedness theory.

\begin{itemize}
\item
  See \citep[ 16]{armbruster2006}
\item
  See \citep{kitay2004}
\end{itemize}

\hypertarget{sociological-neoinstitutionalism}{%
\subsection{Sociological
neoinstitutionalism}\label{sociological-neoinstitutionalism}}

A theory that is systematically drawn upon \citep[ 6-8]{armbruster2006}
is sociological neoinstitutionalism. It is based on the argument that
the belief in efficiency of certain practices or solutions drives
economic actions, rather than the proven efficiency. For example,
\citet[120]{jayatilaka2006} argue that ``{[}i{]}nitiatives to follow
other companies in IS sourcing arrangements could originate from IS and
managerial professionals working in the companies and consultants who
are aware of successful outsourcing arrangements. Mimetic isomorphism
occurs when companies follow the lead of other companies that have
successfully outsourced IT.''

For example, \citet[350]{loh1992} argues that the decision to outsource
their entire IT department to IBM by Eastman Kodak in the late 1980s had
a tremendous impact on other companies' outsourcing decisions.

Mimetic isomorphism is driven by three mechanisms \citep[
150-154]{dimaggio1983}:

\begin{enumerate}
\def\labelenumi{\arabic{enumi}.}
\tightlist
\item
  A coercive authority formally or informally exerts pressure on
  organizations to collude or fall in line. This authority can be
  governmental (through laws \& regulation) and non-governmental
  (through standard operating procedures established at the
  conglomerate-level).
\item
  Uncertainty regarding technology or organizational processes could
  drive organizations to ``model themselves after similar organizations
  in their field that they perceive to be more legitimate or successful.
  The ubiquity of certain kinds of structural arrangements can more
  likely be credited to the universality of mimetic processes than to
  any concrete evidence that the adopted models enhance efficiency.''
\item
  Normative pressures mainly stem from professionalization (infra), as
  the members of an occupation try to establish a ``cognitive base and
  legitimation for their occupational autonomy.''
\end{enumerate}

The result is that large consultancies have been described as carriers,
not only of knowledge, but of legitimacy too. After all, it's their
analyses that validate management decisions.

``Companies that are held internally and externally accountable for how
they `handle' uncertainty will contract consultants as a sign of good
management.'' \citep[ 69]{kieser2006}

\citet[20-21]{zucker1985} argues that this is the result of
trust-building signals (infra) growing beyond their initial goal of
delineating specific expectations. Trust-producing firms (such as
consultancy firms) an sich assume a high status, with the business world
protecting them against failure.

Clearly, this view is mainly appropriated by the critical view since it
raises doubts about the efficient outcomes on the practice of
consultancy.

\hypertarget{signaling-theory}{%
\subsection{Signaling Theory}\label{signaling-theory}}

Another theory that falls in the camp of the critical view is that of
economic signaling theory \citep[ 8-10]{armbruster2006}. Unlike
sociological institutionalism, it treats the economic actors as
experienced and knowledgeable, and not as part of an institution.
Signaling theory argues that in uncertain markets, suppliers invest in
features that signal status, quality and reliability.

\citet[15-16]{zucker1985} argues that these features can also be used to
signal similarity. While nationality, ethnicity and sex can indicate a
common cultural system, or a ``world held in common'', more superficial
(bought or acquired) features can delineate specific expectations in
specific situations. In consulting this translates into degrees,
certificates, using the adequate buzzwords, wearing a suit and driving a
quality car. These indicators signal adherence to the ``rules of the
game.''

Sociological neoinstitutionalism argues that legitimacy-seeking behavior
leads to inefficient market outcomes, while signaling theory argues the
opposite.

\hypertarget{problem-statement}{%
\section{Problem statement}\label{problem-statement}}

Organizations employing the services of a contingent workforce, like
digital consultants, should always be aware of principal-agent problems
arising from information asymmetries. This is caused by the fact that a
transaction between a vendor and a buying organization involves the
delivery of services in the future. As a result, opportunism is always
lurking around the corner.

Two types of opportunism can be identified \citet[242]{clark1993}:

\begin{enumerate}
\def\labelenumi{\arabic{enumi}.}
\tightlist
\item
  Ex ante: Pre-contractual opportunism or adverse selection
\item
  Ex post: Post-contractual opportunism or moral hazard
\end{enumerate}

Hiring consultants for digital services deserves a dedicated scope,
because of the following five properties that make it very prone to
opportunism \citep[ 207]{leslie1995}.

\begin{enumerate}
\def\labelenumi{\arabic{enumi}.}
\tightlist
\item
  Information technology evolves rapidly. Consequently, it involves a
  high degree of uncertainty. (ex ante)
\item
  The underlying economics of IT changes rapidly, making it hard to
  evaluate the consultant's contribution. (ex post)
\item
  IT has penetrated all business functions. It is hard to isolate it
  from organizational functions. The question is rarely about
  ``outsourcing or not'', but more often about ``what tasks will be
  outsourced, and how''. The result is that the responsibilities of
  in-house staff and consultants is very intertwined. (ex post)
\item
  The cost of switching providers are often significant. For example:
  some consultants have exclusivity for implementing a certain solution.
  (ex post)
\item
  Clients might be very inexperienced with regards to IT (and
  outsourcing it). This puts them at a disadvantage for selecting and
  evaluating a consultant (ex ante, ex post).
\end{enumerate}

\citet[59]{aubert1996} also points to a problem of measurement within IT
outsourcing. Contracts often specify all kinds of measures: response
time, uptime, error logs, etc. Although they are linked to explicit
provisions (such as fines, penalties and contract termination), there
are two conditions for them to be effective: (1) observability and (2)
verifiability. The former implies that the client can observe the actual
performance of the agent, while the latter is about verifying
observations and providing evidence.

\hypertarget{adverse-selection}{%
\subsection{Adverse Selection}\label{adverse-selection}}

Adverse selection is associated with the client's inability to determine
the client's capabilities with regards to the assignment. This analogous
to Akerlof's ``Lemons problem'' \citeyearpar{akerlof1970}: due to the
information asymmetry, clients don't want to pay more than the average
price for consultants within a certain niche. While consultants of
below-average quality (``lemons'') benefit from this average price,
above-average consultants will not want to compete, and are crowded out.

\citep[ 69-75]{armbruster2006} outlines various reasons for quality
uncertainty and groups them in two categories.

Category 1: Formal institutional uncertainty

\begin{itemize}
\tightlist
\item
  Consulting is an unbounded profession.
\item
  Consulting is an unbounded industry.
\item
  Consulting has unbounded service lines and product standards.
\end{itemize}

Category 2: Transactional Uncertainty

\begin{itemize}
\tightlist
\item
  Confidentiality
\item
  Product intangibility
\item
  Interdependent cooperation
\end{itemize}

The economic barriers to entry \citep[ 463]{fee2004} in IT consultancy
(and consultancy in general) are few to none. Anyone with experience in
a specific field, sector or technology can wrap it as advice and sell it
to whoever wants to hear it. Furthermore, the author of this paper is
unaware of legal barriers to entry.

According to some, assessing the quality of consultants is impossible.
For example, according do \citet[40]{bloomfield1995}, ``there can be no
presumed separation between technical skills and political skills, and
no ranking between the two in terms of their importance for consultancy
practice and the development of IT in user organizations.'' Furthermore,
\citet[101-102]{bettencourt2002} states that for knowledge-intensive
business services (KIBS) to succeed, a lot depends on the client.
``Client co-production roles {[}\ldots{]} are emergent, multi-faceted,
and highly collaborative because clients themselves possess much of the
knowledge and competence that a KIBS firm needs to successfully deliver
its service solution.''

For evaluating (future) performance of consultants, one has to rely on
informal and relational criteria \citep[ 277]{wright2002}. According to
\citet[250]{clark1993}, ``the main trust-producing mechanism
{[}\ldots{]} is the `closed' social structure; a form of individual
trust. The formal, institutional-based, trust-producing mechanisms are
weak. It is the contractual guarantees, and the history of past
transactions underlying reputation, which overcome the potential effects
of adverse selection and moral hazard.''

Especially in a situation where past transactions are absent,
``management consultancies must convey in some way to their clients that
they have something valuable to offer. {[}\ldots{]} consultants are able
to take control of the process by which impressions and perceptions of
their service are created. By managing the creation of these images
consultants are able to persuade clients of their value and quality.
Management consultancies are therefore `systems of persuasion' \emph{par
excellence} and impression management is not external to the core of
their work but is at its core.'' \citep[ 35]{clark1998}

In the existing literature, these remarks are part of a critical
paradigm regarding consultants \citep[ 4-5]{armbruster2006}. Authors
point to the contestable nature of consulting, the self-interest of
consultancy firms, and the stretching of consultancy advice.

See also:

\begin{itemize}
\tightlist
\item
  \citep{wright2002}
\item
  \citep{david2013}
\item
  \citep{mahoney2016}
\end{itemize}

Another valid point is raised by \citet[23]{basu2011} , who discovered
that pre-qualification efforts displayed little effect on adverse
selection because the consultant might present an excellent reputation,
but individual consultants assigned to a project might lack the required
skills.

\hypertarget{moral-hazard}{%
\subsection{Moral Hazard}\label{moral-hazard}}

See also \citep[ 72-73]{armbruster2006}.

Moral hazard is the result from two elements at the core of transaction
cost economics and agency theory: the ambiguity of the measurement of
individual performance and the goal incompatibility between principal
and agent.

When a particular individual sells his services to another one, it may
be difficult to asses its true value \citep[ 134-135]{ouchi1980}. This
is especially the case when interdependent technologies are involved, as
their implementation and maintenance requires teamwork. On that account,
disentangling individual contribution from the team's joint efforts are
particularly hard. This situation invites opportunism such as slacking
off.

This ``ambiguity of the measurement of individual performance'' \citep[
135]{ouchi1980} is palpable in digital consultancy.

\hypertarget{corporate-governance-management-control-systems}{%
\subsection{Corporate Governance \& Management Control
Systems}\label{corporate-governance-management-control-systems}}

In the existing literature, several governance mechanisms have been
proposed. Each of them is somehow related to trust\footnote{Trust is
  defined in the broad sense: ``the willingness of a party to be
  vulnerable to the actions of another party based on the expectation
  that the other will perform a particular action important to the
  trustor, irrespective of the ability to monitor or control that other
  party.'' \citep{kee1970} If the level of a perceived risk is bigger
  than the level of trust, the trustor will not engage in a risk-taking
  relationship.} between the business partners. Either they increase
trust, or they eliminate the need for trust. From a transaction cost
economics perspective, trust makes transactions cheaper and allows for
greater flexibility, as it requires less incentivization, and
specification and monitoring of contracts {[}@nooteboom1996 989{]}.

\citet[265]{liberatore2010} found that building trust, and goal
congruence, can help solve the agency problem because it drives the
consultancy firm's motivation for short-term profits towards long-term
business and reputation. However, trust is not unbounded and should not
be taken for granted as it might break down \citep[ 988]{nooteboom1996}.
Golden opportunities that require defection are always lurking and even
tempting for the most trustworthy.

\citet[193-194]{kirilov2012} breaks trust factors down in trust-building
and trust-sustaining factors. The former is about signaling ability
through references, experience and reputation. The latter is about
signaling integrity: effective and transparent communication,
proactivity, monitoring and consistently meeting contractual
obligations. In that sense, building trust can be describe as a
``Bayesian-like decision process'' in which all trust-relevant
information is carefully scrutinized: the proportion of ``cooperative
choices or long-term behavioral patterns''. \citep[ 995]{lewicki2006}

\citet{williams1988} describes two dimensions to position `trust': the
context and egotism. This yields four sources of trust and cooperation.

\begin{longtable}[]{@{}
  >{\raggedright\arraybackslash}p{(\columnwidth - 4\tabcolsep) * \real{0.2466}}
  >{\raggedright\arraybackslash}p{(\columnwidth - 4\tabcolsep) * \real{0.3425}}
  >{\raggedright\arraybackslash}p{(\columnwidth - 4\tabcolsep) * \real{0.4110}}@{}}
\toprule\noalign{}
\begin{minipage}[b]{\linewidth}\raggedright
\end{minipage} & \begin{minipage}[b]{\linewidth}\raggedright
Macro
\end{minipage} & \begin{minipage}[b]{\linewidth}\raggedright
\textbf{Micro}
\end{minipage} \\
\midrule\noalign{}
\endhead
\bottomrule\noalign{}
\endlastfoot
Egotistic & Coercion or fear of sanctions from authorities & Material
advantage \\
Non-egotistic & Ethics such as values and norms of proper conduct &
Bonds of friendship, kinship of a certain degree of empathy \\
\end{longtable}

As we'll see below, governance mechanisms can be designed to foster each
of these four grounds for cooperation.

\citet[366-374]{shapiro1992} identified three bases of trust. Investing
in either of these can be matched to the level and benefits of the trust
desired, and the costs and risks associated with it. The three bases of
trust are the following.

\begin{enumerate}
\def\labelenumi{\arabic{enumi}.}
\tightlist
\item
  Deterrence-based trust -- or ``calculus-based trust'' \citep[
  119]{lewicki1996} -- results in predictive behavior of the agent
  because there are measures in place to prevent hostile actions. These
  measures generate a potential cost for acting in a distrustful way,
  outweighing the advantages of doing so.
\item
  Knowledge-based trust emerges from prior contacts based on the premise
  that through ongoing interaction, partners get to know each other
  better and develop trust around norms of equity \citep[
  92]{gulati1995}. People act cooperatively when they expect their
  partner to behave in the same way and reciprocate. Each has something
  to give the other. Small gestures, such as a consultant providing
  market insights, a client inviting consultants to team buildings or
  company events, pressure each other into conformity \citep[
  63]{macaulay1963}.
\item
  Identification-based trust rests on the premise that when both parties
  share the same preferences they tend to behave in a more trustworthy
  manner towards each other \citep[ 371]{shapiro1992}. Several
  conditions can build trust based on identification, of which joint
  products or goals, a common team name, proximity and shared values are
  mentioned explicitly in the research. An interesting finding by
  \citet[408]{schoenherr2015} is that process integration not only
  serves as a mechanism for information exchange, but also for
  engendering reciprocity, yielding greater levels of trust.
\end{enumerate}

According to \citet[1011]{lewicki2006}, the bases for trust change over
time. For example, as parties get to know each other, their trust
evolves from deterrence-based to knowledge-based. However, rarely does a
transition occur from knowledge-based trust to identification-based
trust, as it requires identification with each other and the development
of strong affect between two parties.

This is highly compatible with the research by
\citet[717-720]{mayer1995}, which describes three groups of trust
antecedents: ability, integrity and benevolence.

\begin{itemize}
\tightlist
\item
  \emph{Ability} describes the skills, competencies and characteristics
  that enable a party to have influence within a specific domain.
\item
  \emph{Integrity} involves the trustor's perception that the trustee
  adheres to the principles that the trustor finds acceptable for that
  domain.
\item
  \emph{Benevolence} is the extent to which the trustor believes that
  the trustee wants to do good to the trustor, and looks beyond their
  profit motive.
\end{itemize}

Finally, each of these antecedents maps rather well to the three trust
definitions outlined by \citet[37-40]{sako1992}: competence trust,
contractual trust and goodwill trust.

Below, a fairly exhaustive list of proposed governance measures is
outlined and grouped by their corresponding trust antecedent.

See also \citet{lewicki2006} and \citet{kirilov2012}.

Also: how does `governance' relate to control mechanisms? See
\citet{smith2003}.

\hypertarget{ability}{%
\subsubsection{Ability}\label{ability}}

\hypertarget{reputation}{%
\paragraph{Reputation}\label{reputation}}

\citet[193]{kirilov2012} describes reputation as ``a mixture of the
brand name of the enterprise, executive management background, maturity
level, customer references, and independent quality assessments.''

\citet[75-76]{armbruster2006} distinguishes three types of reputation:

\begin{enumerate}
\def\labelenumi{\arabic{enumi}.}
\tightlist
\item
  Public reputation is the perception of a consulting firm's (or
  individual consultant) past performance or potential. For this reason,
  large consultancies make it known that they spend a lot of resources
  on the search and selection of recruits. \citep[ 91]{kieser2006} While
  there are few to no barriers to enter the market as a whole with a
  newly-found consulting firm, public reputation is a huge barrier to
  reaching its upper end. Public reputation is like a public good ; the
  information is non-excludable and non-rivalrous.
\item
  Experience-based trust relates to personal experience with a specific
  partner. A positive relation drives future action because a partner is
  less likely to act distrustfully in one transaction if future benefits
  are jeopardized. \citep[ 367]{shapiro1992} However, trust evolves
  slowly, and maintaining it requires commitment. That's why it is often
  constrained to a small group of business partners.
\item
  Networked reputation is a firm or consultant's reputation within a
  network of business relations \citep[ 271]{gluckler2003}. For example,
  ``if a trusted party cannot provide the resources that are needed,
  their relations can be used in order to obtain trustworthy information
  about parties one is not connected to.'' \citep[ 280]{gluckler2003}
  Not only consulting partners can make recommendations;
  \citet[308]{honer2006} proposes building informal networks that would
  allow managers to communicate their experience in consulting to each
  other.
\end{enumerate}

Public reputation has a high market scope, as all potential partners are
active in a specific market, but it generally results in less certainty,
since public reputation is easily manipulated. Experience-based trust
produces high certainty, but a low market scope, since rarely, one has
had interpersonal experience with all potential partners. Finally,
networked reputation sits somewhere in between: via a social network,
one can estimate the reputation of various consulting partners and it
also produces high certainty, since relations are at stake.

\citet[243-244]{clark1993} asked 55 respondents about the factors that
are important when choosing an executive search \& selection
consultancy. ``Reputation of individual consultants'' and ``reputation
of the consultancy'' are in the top three factors. This reputation often
arises from ``a history of past transactions with individual
consultants. Frequent transactions between consultants and clients leads
to familiarity which underpins the latters' assessment of the former.''
In other words, because finding a new consultants implies a search cost
\citep[ 1072]{wilson2012}, incumbent consultants are expected to receive
new contracts as long as the cost incurred from a potential sub-optimal
performance is lower than the search cost of finding a new consultant.

The findings in \citet{clark1993} are confirmed by
\citet[285]{richter2009} who found, with regards to projects that
involve client-specific information, that ``clients are willing to
involve external consultants with whom they have established a
relationship of trust in the execution of such projects. {[}\ldots{]}
intermediate forms of governance between the extremes of market
procurement from an anonymous provider and fully-fledged integration,
are not only viable, but an effective option for clients to procure
managerial services.'' Organizations are more inclined to work with
consultants with whom they have no experience when no client-specific
information or industry expertise is required. A prior client/supplier
working relationship is also listed as an important determinant in the
meta-analysis by \citet[235]{lacity2011}.

According to \citet[516]{nayyar1990} ``reputation performs as an
implicit contract. It is enforced by the seller's concerns about future
demand for the service provided. {[}\ldots{]} reputation is likely to
exhibit characteristics of a public good. Once acquired, it can be user
over and over again in the context of other services or markets.''

According to a survey with 150 German companies \citep[ 91]{kieser2006},
for 73\% of respondents, reputation is a deciding factor. Furthermore,
50\% of respondents indicated that recommendations from managers of
other companies is a deciding factor.

\hypertarget{procurement}{%
\paragraph{Procurement}\label{procurement}}

``Management consulting is a highly interactive service in which
interpersonal trust and `liking' play a central role. {[}It is often
described{]} as very similar to recruiting an employee.'' \citep[
185]{furusten2005} Experienced managers buy the services of individuals
in whom they have confidence, not from consulting companies. The role if
interpersonal is motivated by the argument that the need for consultants
is often not driven by the organization, but by the manager's personal
needs and insecurities. What is results is the assumption that managers
do not always behave in the company's interest when dealing with
consultants, and governance measures are required \citep[
300]{honer2006}. This ties in to the ``agent's agent'' argument in
\citet{fincham2002}, discussed earlier.

\citet[307]{honer2006} describes how a central purchasing/project office
which selects consultants on behalf of the managers could streamline the
hiring process of consultants. Furthermore, this office could be tasks
with control and coordination of all consulting projects, going beyond
mere selection. Nevertheless, this could be incongruent with company
culture and managerial budget responsibility.

Typically, the involvement of procurement intermediaries
commodify\footnote{Commodification is ``the process whereby an object
  (in the widest sense of the term, meaning a thing, an idea, a
  creature, etc.) comes to be provided through, and/or represented in
  terms of, a market transaction'', crystalizing their value into a
  price. \citep{carvalho2008}{]}} management knowledge \citep[
205-206]{omahoney2013}. However, commodification is often resisted on
two fronts, both representing a power struggle.

\begin{itemize}
\item
  On the one hand, the ``producers'' of management knowledge, (i.e.~the
  consultancy firms) are well aware that commodification could
  potentially neglect their competitive advantages, having a deleterious
  effect on their profit margins.
\item
  On the other hand, as procurers define the problem of management, the
  come in direct competition with the managers they represent.
\end{itemize}

\citet[305-306]{honer2006} found that friction between managers and
procurement can result in three behavioral patterns:

\begin{itemize}
\item
  departmentalism: managers claiming that their department, contrary to
  the company as a whole, is very transparent regarding their use of
  consultants.
\item
  authority protection: managers claiming that intervention ins their
  authority is neither necessary, nor desired and that they have
  authority to how their department's problems are solves.
\item
  laziness: managers do not recognize that using consultants effectively
  is a significant managerial task, and underreport unsuccessful
  projects.
\end{itemize}

\hypertarget{third-party-assessment}{%
\paragraph{Third-Party Assessment}\label{third-party-assessment}}

\citet[57-62]{zucker1985} describes the rise of the ``social overhead
sector'' in the 20th century. This sector acts as an ``intermediary'' in
a variety of situations: stock brokers, real estate agents, banks, etc.
The same principle can be applied to consultants: assessment by a
third-party agency can prevent adverse selection.

See \citep[ 76-77]{armbruster2006}.

\hypertarget{integrity}{%
\subsubsection{Integrity}\label{integrity}}

\hypertarget{monitoring}{%
\paragraph{Monitoring}\label{monitoring}}

``At the post-contractual stage, agency theory asserts that monitoring
the agent gathers information about the agent and helps reduce
opportunism. Monitoring places an uncomfortable social pressure on the
agent that increases compliance. It also increases the principal's
ability to detect the agent's opportunism and thus to appropriately
reward or sanction agent behavior. It reduces the agent's motivation to
justify a failed strategy, and promotes actions consistent with
shareholder goals.'' \citep[ 13]{basu2011}

Researchers have emphasized three broad ways in which consultants should
be monitored \citep[ 15]{basu2011}:

\begin{enumerate}
\def\labelenumi{\arabic{enumi}.}
\tightlist
\item
  When the consultancy firm gives their agreement to the specified
  deliverables and accompanying deadlines.
\item
  During the implementation of a project, the client verifies that the
  deliverables are being produces according to the original plan by
  thoroughly and regularly assessing reviews and written and oral
  progress reports. Meeting with consultants is essential to ensure that
  consultants share all relevant information in a timely manner.
\item
  During the implementation of a project, the client checks that the
  consultants do not sacrifice quality nor scope to meet deadlines.
  Furthermore, the originally committed staff should not be changed
  without approval.
\end{enumerate}

Nevertheless, since monitoring often rely on surrogate measures,
consultants can displace their behavior toward these surrogates in order
to appear to be behaving well. \citep[ 281]{shapiro2005}

See also @nooteboom2000.

\hypertarget{contractual-obligations-vs.-contract-flexibility}{%
\paragraph{Contractual Obligations vs.~Contract
Flexibility}\label{contractual-obligations-vs.-contract-flexibility}}

Contracts have several functions \citep[ 924]{noteboom1996}: a legal
document to constrain opportunism, a record of agreement to guide
technical coordination, prevention of misunderstanding and even a
ritualistic function to seal the intention to cooperate. However, not
all future contingencies are known, the cost of setting up a detailed
contract and monitoring it can be significant (supra), and contracts may
form a straight-jacket that blocks the utilization of future
opportunities. Finally, detailed contracts can shroud a contract in an
atmosphere of mistrust, derailing the relation before it has even
started.

According to \citet[924]{nooteboom1996}, most inter-firm cooperation is
described in a contract of some form. However, ``{[}t{]}he question is
not so much whether there is a contract, but whats its content is an how
elaborate it is.''

\citet{mcfarlan1995} claim that it is important to have flexibility in
an outsourcing contract, because the target state of a project might
change due to evolving technology and business environment.
Nevertheless\ldots{}

\citet[4]{lacity2012} describes that there is substantial evidence that
positive outsourcing outcomes are associated with:

\begin{itemize}
\tightlist
\item
  more detailed contracts with regards to scope, service levels,
  responsibilities and adaption to change;
\item
  shorter-term contracts;
\item
  high-value contracts.
\end{itemize}

More detailed contracts, when resulting in requirements uncertainty, is
an enabler of goal congruence and trust between the consultancy firm and
the client, which is found to result in a better project performance
\citep[ 264]{liberatore2010}.

\hypertarget{purchasing-regulation}{%
\paragraph{Purchasing Regulation}\label{purchasing-regulation}}

See \citet[4-5]{sturdy2021}

\citet[307]{honer2006} proposes setting op standardized processes for
dealing with consultants. These would give managers ``clear instructions
for dealing with consultancy and enhance the principal's control.''
However, the author highlights that this might impact a manager's
perception of their autonomy and they might act in ways to ignore or
bypass these rules.

\hypertarget{whistleblowing}{%
\paragraph{Whistleblowing}\label{whistleblowing}}

Also media.

\hypertarget{incentivization}{%
\paragraph{Incentivization}\label{incentivization}}

Contrary to the design and enforcement of contractual obligations,
management by self-interest (i.e.~incentivization) ``has the advantage
that it is cheaper than contracts, is more flexible, and it is in the
players' own interest to be seen to comply with agreements.'' \citep[
924]{nooteboom2000} However, it's not a silver bullet as it requires a
need for observation, measurement and monitoring. Yet, ``How does one
measure and monitor degree of dependence, spillovers, and specificity of
investments?'' It's hard to quantify compensation for such intangible
risks. Furthermore, it is hard to maintain, since competences and
external conditions are constantly changing.

``Basing the agent's rewards and incentives on imperfect surrogates of
performance leads to moral hazard, but aligning the preferences of the
agent and the principal through an appropriate reward structure helps
curb the agent's opportunistic behavior.'' \citep[ 13-15]{basu2011}
Several actions are proposed to align incentives: link payment to
completion of the promised deliverables, sharing of cost savings or
overruns with the consultancy firm, incentives and penalties related to
timely completion of a project.

\citet[264-266]{liberatore2010} finds that a higher goal congruence
between the consultancy firm and the client is an enabler of project
performance.

A particular form of incentivization is ``hostage-taking'': ``one-sided
ownership of specific assets may be balanced by one-sided hostages going
the other way, or by a rigorous reputation mechanism.'' \citep[
924]{nooteboom2000} When there are no trust-related nor legal
enforcement mechanisms, parties can be discouraged from forming
long-term relationships. According to \citet[47-48]{werner1993},
``hostages'' are used in situations where rational behavior would lead
to sub-optimal outcome, in the Paretoian sense. In game theoretical
terms, hostage-taking is used to prevent defecting behavior.

In terms of the subject of this paper, ``hostages'' could come in the
form of contingent fees. See \citep[ 243]{clark1993}

See \citet{tosi1997}.

Reputation could also be taken hostage. \citep[ 368]{shapiro1992}

\hypertarget{clan-mechanisms}{%
\paragraph{Clan Mechanisms}\label{clan-mechanisms}}

\citet{ouchi1980} proposes to provent opportunism through the
establishment of a ``clan''\footnote{\citet{ouchi1980} borrows the
  concept of a clan from \citet[127]{durkheim1997} who gives the term
  `clan' to ``a horde that has ceased to be independent and has become
  an element in a more extensive group {[}\ldots{]} It is a family in
  the sense that all the members who go to make it up consider
  themselves kin to one another {[}\ldots{]} The affinities produced by
  sharing a blood kinship are mainly what keeps them united. What is
  more, they sustain mutual relationships that might be termed domestic,
  since these are to be found elsewhere in societies whose family
  character is undisputed: I mean collective revenge, collective
  responsibility and, as soon as individual property makes an
  appearance, mutual heredity.''}, which involves commitment from all
parties and eliminates short-term inequities over time. Clans achieve
the ``union of objectives between individuals which stems from their
necessary dependence upon one another. {[}\ldots{]} {[}c{]}lans display
a high degree of discipline {[}\ldots{]} achieved through an extreme
form of the belief that individual interests are best served by a
complete immersion of each individual in the interests of the whole.''

Clans differ from bureaucracies and markets in that they don't require
auditing or evaluation as it takes place ``through the kind of subtle
reading of signals that is possible among intimate coworkers but which
cannot be translated into explicit, verifiable measures.'' \citep[
137]{ouchi1980} For clans to succeed, the following conditions are
required: reciprocity, legitimate authority, common values/beliefs and
traditions. The former three are normative, while the latter is
informal. When these conditions are not met, clans are merely ceremonial
and ritualistic.

Within the context of digital consultancy, reciprocity is important in
the sense that everybody within a team, both employees and consultants
needs to have the freedom to call each other out. Furthermore, managers,
product owners and lead developers can assume the role of legitimate
authority.

However, both the conditions of traditions and common values/beliefs are
the hard to achieve. Mayo already construed in 1945 that ``by reason of
external circumstance {[}\ldots{]} groups {[}that{]} have little
opportunity to form, the immediate symptom is labor turnover,
absenteeism, and the like.'' \citep[ 111]{mayo1945} A partial
socialization may be accompanied by market or bureaucratic mechanisms
\citep[ 138]{ouchi1980}, such as the participation in company meetings,
to achieve common values and beliefs.

\hypertarget{third-party-moderation}{%
\paragraph{Third-Party Moderation}\label{third-party-moderation}}

\citet[6-7]{babin2017} describes a single case study in which trust in
an outsourcing relationship had eroded over time. Instead of parting
ways, it was clear that both parties needed each other. A proactive
approach in the form of a third-party facilitated workshop successfully
addressed the trust problem. ``When both sides feel hurt, and do not
trust the other side, it can be difficult to take preliminary steps to
begin to repair inter-organizational trust. An objective outsider using
a formal approach proved useful to initiating the trust repair
activities.

\hypertarget{first-party-assessment}{%
\paragraph{First-Party Assessment}\label{first-party-assessment}}

As \citet[111]{mayo1945} describes: ``the belief that the behavior of an
individual within the factory can be predicted before employment upon
the basis of a laborious and minute examination by tests of his
technical and other capacities is mainly, if not wholly, mistaken.
Examination of his developed social skills and his general adaptability
might give better results.''

``A second vehicle for improving understanding and predictability
between partners is to conduct very good research on the potential
partner before a relationship is engaged. This research is directed at
assessing the real compatibility, or''interpersonal fit,'' between
partners, as well as assessing the degree to which a potential partner
engages in predictable behavior.'' \citep[ 370]{shapiro1992}

\hypertarget{consulting-database}{%
\paragraph{Consulting Database}\label{consulting-database}}

\citet[5]{sturdy2021} proposes internalization of supplier information.
For example, \citet{mohe2006} set up an \emph{infobase} in
DaimlerChrysler that can be used by internal managers for managing
consultants and consulting projects. It contains news and trends
regarding consulting, process guidelines, consultant profiles, past
projects and management info (such as price).

\citet[308]{honer2006} proposes an institution within an organization
that has the staff and capabilities to support managers in dealing with
consultancy. This institution would hold data on previous consultancy
engagements, assisting managers in choosing the adequate consultancy
partner. Their service is entirely voluntary and secures discrete
conduct and sensitivity when dealing with confidential information. This
could also be slimmed down

\hypertarget{benevolence}{%
\subsubsection{Benevolence}\label{benevolence}}

\hypertarget{cultural-understanding}{%
\paragraph{Cultural Understanding}\label{cultural-understanding}}

Cultural understanding

\hypertarget{psychological-contract-obligations}{%
\paragraph{Psychological Contract
Obligations}\label{psychological-contract-obligations}}

According to \citet[357]{ang2004}, the legal interpretation of an IT
outsourcing contract is too limited. Instead, they claim that the
construct of a \emph{psychological contract} is more appropriate for
analyzing the relationship between an IT service supplier and customer.
The concept of a psychological contract states that contracts are
``idiosyncratically perceived and understood by individuals {[}and{]}
{[}s{]}ubjectivity in the contract leads to disagreement between
parties.'' \citep[ 21]{rousseau1993}

Consequently, the psychological contract not only comprises the legal
contract, but also the unwritten promises, interpersonal relations, and
the individual interpretations and perceptions. Since consultancy
contracts can become extremely complex (with project descriptions going
into the ten thousands of words), and the involved parties entangled in
multiple ways, these intangible aspects can gain prominence. The
research in \citet[369-70]{ang2004} outlines several psychological
contract obligations that positively impact the success of an outsourced
IT project.

\begin{itemize}
\tightlist
\item
  On the supplier side: (1) clear authority structures, (2) knowledge
  transfer by educating the customer, (3) building inter-organizational
  teams.
\item
  On the customer side: (1) clear specification of requirements, (2)
  prompt payment, and (3) project ownership and monitoring.
\end{itemize}

The strength of psychological contract theory is threefold:

\begin{enumerate}
\def\labelenumi{\arabic{enumi}.}
\tightlist
\item
  it focuses on mutual obligations;
\item
  the emphasis is on psychological obligations;
\item
  the emphasis is on the individual level--not on the organizations as
  parties of the contract.
\end{enumerate}

Closely related is the work by \citet[9-13]{willcockskern} that makes a
distinction between the contractual level and the cooperative level. The
contractual level is about payment for the exchange of services and the
transfer of assets, information \& consultants. The cooperative level
involves formal communication mechanisms; personal investments in time,
resources \& knowledge; mutual goals \& objectives and social bonds. The
atmosphere surrounding the former is heavily impacted by developments at
the latter. A respondent in \citet[9]{willcockskern} states that ``the
contract is a bit like a nuclear deterrent. You need one and you have
got to have a framework, but if you've got to use it you are probably in
trouble.''

\hypertarget{relational-contracts}{%
\paragraph{Relational Contracts}\label{relational-contracts}}

\citet[10-12]{rousseau1993} describes a continuum between transactional
contracts and relational contracts. On one end of the continuum, the
former is based on ``short-term monetizable agreements with limited
involvement of each party in the lives and activities of the other.'' On
the other end, the latter describes ``agreements based upon exchanges of
both socioeconomical and monetizable elements, duration which is
open-ended and often long term, and a high degree of flexibility.''

\hypertarget{beyond-controls}{%
\subsubsection{Beyond Controls}\label{beyond-controls}}

\hypertarget{professionalization}{%
\paragraph{Professionalization}\label{professionalization}}

Professionalization is a mechanism that typically protects clients
against self-proclaimed experts but who are unqualified and expose their
clients to even greater risks than the ones they are supposed to advice
on. \citep[ 71]{kieser2006} Furthermore, they also protect experts
against colleagues that could ruin the reputation of the whole industry.
Finally, professionalization typically results in the establishment of
an ethos, which fosters\footnote{A professional ethos is ``a set of
  written and unwritten rules that guide professional practice.''
  \citep{enstad2017}} trust between a profession and the public at large
\citep{sokolowski1991} Through these mechanisms, ``professions are
social devices to limit agency costs.'' \citep[ 276]{shapiro2005}

Certain countries, like Germany, have made attempts to professionalize
the sector through the establishment of specialized university programs
to make the use of the title ``consultant'' dependent on the attainment
of specific qualifications. These attempts haven't turned out to be
successful: today, consultancy is not yet a profession, not in the
classical sense. \citep[ 73]{kieser2006}.

According to a popular functionalist view \citep{goode1957}, which
identifies criteria that distinguish professions from occupations: (1)
members share a common identity, (2) it's a life-long calling, (3) share
common ideals, (4) have a shared self-conception, (5) behave distinctly
towards non-members, (6) use specific language, (7) easy to recognize
and (8) follow an ethical code by a powerful association.

The power approach, on the other hand, is much different from the
functionalist approach. It argues that professions are a group of people
that wants to achieve expert status within society, attaining certain
privileges. \citep[ 75]{kieser2006}

Both are potent frameworks for providing arguments why the practice of
consultancy is not likely to professionalize. From the functionalist
perspective, we identify the following:

\begin{itemize}
\item
  Consultants do not seek to differentiate themselves from their
  clients, \emph{au contraire}, they seek partnership, as to truly
  understand the problems and opportunities a client faces.
  \citep{fincham2006}
\item
  A large part of the \emph{raison d'être} of consultants relates to
  their capability of transferring expertise to the clients. This
  differs from professionals, who are approached by clients to solve a
  problem, without the expectation of knowledge transfer.
  \citep{oakley1993}
\item
  Consulting organizations are extremely diverse, both in terms of
  subfields (IT, HR, strategy, \ldots) and in terms of organization
  size, structure and business models. \citep[ 79, 89]{kieser2006}
\end{itemize}

The arguments of the power approach are the following:

\begin{itemize}
\item
  On the one hand, there are the small consultancies which could try to
  establish associations to build up their reputation. On the other
  hand, there are the big consultancies, with a high reputation, that
  only stand to lose because of increased competition with (a priori)
  lower reputation organizations. For this reason, the probability of
  establishing all-encompassing associations is rather low \citep[
  77]{kieser2006}.
\item
  Many employees of large consultancy firms see their tenure as a
  platform to self-employed consultant or partnership at a smaller
  consultant. A title, or membership of an association would facilitate
  further split-offs at large consultancies \citep[ 80]{kieser2006}.
\end{itemize}

One could even argue that professionalization is not even relevant in
today's capitalism. The ``individual given class status, autonomy,
social elevation, in return for safeguarding our well-being and applying
their professional judgement on the basis of a benign moral or cultural
code {[}\ldots{]} no longer exists.'' \citep[1-2]{dent2013} Trust and
respect in today's professional is ``earned through their ability to
perform to an externally given set of performance indicators.'' In that
sense, ``consultants are the real champions of creating a new
professional \emph{appearance}.'' \citep[ 95]{kieser2006} By combining a
promise of market success and professionalism they are ``market
professionals''.

\hypertarget{regulation}{%
\paragraph{Regulation}\label{regulation}}

\begin{itemize}
\tightlist
\item
  Three sources of regulation can be identified (\citet{clark1993}
  246-247).
\item
  See \citet[813-817]{muzio2011}.
\end{itemize}

While in some countries, like Austria, there is a registration system
for consultants, governments have been ambivalent towards professional
regulation of the consulting sector \citep[ 814]{muzio2011}. According
to \citet{leicht2006}, this is the result of (Western) governments
embrace of market logic and their reluctance to intervene with
``additional coercive pressures in the wake of questionable professional
conduct.''

One important antecedent for regulation is professionalization of the
sector, which is highly unlikely in the current institutional
environment. Although ``expertise is increasingly more important in
differentiating societies, it is difficult to defend the status of
professions, to say nothing of establishing new ones. The trend is
towards deregulation and deprofessionalization of existing professions,
rather than towards establishing new ones.'' \citep[ 90]{kieser2006}

Finally, governments are often large consumers of consultancy services,
making them wary of supporting regulation of the sector, as it could
drive up prices and reduce their autonomy in hiring them \citep[
815]{muzio2011}.

\hypertarget{government-initiated-codes}{%
\paragraph{Government Initiated
Codes}\label{government-initiated-codes}}

See \citet[3-4]{sturdy2021}

\hypertarget{self-imposed-sectoral-codes}{%
\paragraph{Self-imposed Sectoral
Codes}\label{self-imposed-sectoral-codes}}

See \citet[4]{sturdy2021}

\hypertarget{specific-commitment}{%
\paragraph{Specific Commitment}\label{specific-commitment}}

See \citet[12]{sturdy2021}

\hypertarget{control-configurations}{%
\subsubsection{Control configurations}\label{control-configurations}}

See \citet[286-289]{smith2003}

\hypertarget{research-questions}{%
\section{Research Questions}\label{research-questions}}

Abstractie maken van zaken zoals cultuur, bedrijfsgrootte, etc. Link met
HR \& internaliseren van externen, externe kennis.

1a. Why do Belgian firms rely on digital consultants? 1b. What are
inhibitors \& enablers for success of digital consultants? 1c.
Definition of success

\begin{enumerate}
\def\labelenumi{\arabic{enumi}.}
\setcounter{enumi}{1}
\tightlist
\item
  Do Belgian firms see PA problems with digital consultants?
\item
  Which control mechanisms do Belgium firms have in place with regards
  to adverse selection and moral hazard of digital consultants?
\item
  Which control mechanisms positively impact success of engaging with a
  consultancy firm?
\end{enumerate}

Research questions for side projects:

\begin{enumerate}
\def\labelenumi{\arabic{enumi}.}
\tightlist
\item
  Why do people join a consultancy firm or become an independent
  consultant?
\item
  Do reputational effects exist on the individual consultant level or on
  the firm level?
\end{enumerate}

\bibliographystyle{agsm}
\bibliography{references.bib}


\end{document}
